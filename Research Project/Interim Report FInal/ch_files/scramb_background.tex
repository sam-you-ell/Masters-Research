\section{Background Theory}
\subsection{Quantum Information and Computing}
\vspace{-0.15in}
Quantum Information and Computation is built upon the concept of quantum bits (qubits for short), represented as a linear
combination of states in the standard basis,
$|\psi\rangle = \alpha |0\rangle + \beta |1\rangle$, where $\alpha, \beta$ are complex probability amplitudes
and the vectors, $|0\rangle$, $|1\rangle$ are the computational basis states that form the standard basis.

This is extended to multi-party or composite systems of $n$ qubits via the tensor product. The total Hilbert space, $\cal{H}$ of
a many-body system is defined as the tensor product of $n$ subsystem Hilbert Spaces,
\begin{equation}
    {\cal{H}} = \otimes_n {\cal{H}}_n = {\cal{H}}_{1} \otimes {\cal{H}}_{2} \otimes \dots \otimes {\cal{H}}_n.
\end{equation}
The computational basis states of this system, are tensor products of qubit states, often written as a string \cite{schumacher_westmoreland_2010},
\[|x_1\rangle \otimes |x_2\rangle \otimes ... \otimes |x_n\rangle \equiv |x_1 x_2... x_n \rangle. \]

The evolution and dynamics of many-body systems can be represented via quantum circuits, constructed from a set of quantum logic gates acting
upon the qubits of a system. Analogous to a classical computer which is comprised of logic gates that act upon
bit-strings of information. In contrast, quantum logic gates are linear operators acting
on qubits, often represented in matrix form. This allows for the decomposition of a unitary evolution into a
sequence of linear transformations. A common practice is to create
diagrams of such evolutions, with each quantum gate having their own symbol, analogous to circuit diagrams in classical
computation, allowing the creation of complicated quantum circuitry that can be directly mapped to a sequence of linear
operators acting on a one or more qubits. Some example gate symbols can be seen in Fig. \ref{Paulis}.


\begin{figure}[h]
    \centering
    \begin{subfloat}[pauliX]{
        \centering
        \begin{quantikz}
            &  \gate{X} 
                &  \qw
        \end{quantikz}
    }
    \end{subfloat}
    \hspace{10pt} 
    \begin{subfloat}[pauliY]{
        \centering
        \begin{quantikz}
            & \gate{Y}
                & \qw
        \end{quantikz}
    }
    \end{subfloat}
    \hspace{10pt} 
    \begin{subfloat}[pauliZ]{
        \centering
        \begin{quantikz}
            & \gate{Z}
                & \qw
        \end{quantikz}
    }
    \end{subfloat}
\end{figure}

The gates shown in Fig. \ref{Paulis} are known as the Pauli operators, equivalent
to the set of Pauli matrices, $P \equiv \{X, Y, Z\}$ for which $X, Y \text{ and } Z$ are defined in their matrix representation as
\begin{align}
    \label{PauliMatrices}
    X = \begin{bmatrix}
            0 & 1 \\
            1 & 0
        \end{bmatrix},
     &  &
    Y = \begin{bmatrix}
            0  & i \\
            -i & 0
        \end{bmatrix},
     &  &
    Z = \begin{bmatrix}
            1 & 0  \\
            0 & -1
        \end{bmatrix},
\end{align}

These gates are all one-qubit gates, as they only act upon a single qubit.
Together with the Identity operator, $I$, the Pauli matrices form an algebra .
Satisfying the following relations
\begin{align}
    XY = iZ,  &  & YZ = iX,  &  & ZX = iY,  \\
    YX = -iZ, &  & ZY = -iX, &  & XZ = -iY,
\end{align}
\begin{align}
    X^2 = Y^2 = Z^2 = I.
\end{align}

Notably, the set of Pauli matrices and the identity form
the Pauli group, ${\cal P}_n$, defined as the $4^n$ $n$-qubit tensor products of the Pauli matrices (\ref{PauliMatrices}) and the
Identity matrix, $I$, with multiplicative factors, $\pm 1$ and $\pm i$ to ensure a legitimate group is formed.
For clarity, consider the Pauli group on 1-qubit, ${\cal P}_1$;
\begin{equation}
    {\cal P}_1 \equiv \{ \pm I, \pm iI, \pm X, \pm iX \pm Y, \pm iY, \pm Z, \pm iZ\},
\end{equation}
From this, another group of interest can be defined, namely the Clifford group, ${\cal C}_n$, defined as a
subset of unitary operators that normalise the Pauli group \cite{orthogonalcodes} *INSERT CLIFFORD GROUP DEFINITION*.
The elements of this group are the
Hadamard, Controlled-Not and Phase operators.

The Hadamard, $H$, maps computational basis states to a superposition of computational basis states,
written explicitly in it's action;
\begin{align*}
    H |0\rangle = \frac{|0\rangle + |1\rangle}{\sqrt{2}}, \\
    H |1\rangle = \frac{|0\rangle - |1\rangle}{\sqrt{2}},
\end{align*}
or in matrix form;
\begin{equation*}
    H = \frac{1}{\sqrt{2}} \begin{bmatrix}
        1 & 1  \\
        1 & -1
    \end{bmatrix}
\end{equation*}
Controlled-Not ($CNOT$) is a two-qubit gate. One qubit acts as a control for an operation to be perfomed
on the other qubit.
It's matrix representation is
\begin{align}
    \label{CNOT}
    CNOT =
    \renewcommand{\arraystretch}{0.75}
    \begin{bmatrix}
        1 & 0 & 0 & 0 \\
        0 & 1 & 0 & 0 \\
        0 & 0 & 0 & 1 \\
        0 & 0 & 1 & 0
    \end{bmatrix}.
\end{align}
The Phase operator, denoted $S$ is defined as,
\begin{align*}
    S =
    \begin{bmatrix}
        1 & 0                  \\
        0 & e^{i\frac{\pi}{2}}
    \end{bmatrix}.
\end{align*}

The CNOT operator is often used to an generate entangled state. One such state is the maximally entangled 2-qubit state,
called a Bell state, $|{\bm\Phi}^{+}\rangle_ = (|00\rangle + |11\rangle)/\sqrt{2}$. This is prepared
from a $|00\rangle$ state, by applying a Hadamard to the first qubit, and subsequently a Controlled-Not gate
\begin{align}
    H \otimes I |00\rangle = \left (\frac{|0\rangle + |1\rangle }{\sqrt{2}}\right )|0\rangle \\
    CNOT \left (\frac{|0\rangle + |1\rangle }{\sqrt{2}}\right )|0\rangle = \frac{|00\rangle + |11\rangle}{\sqrt{2}}
\end{align}
The corresponding circuit representation of this preparation is given in Fig. \ref{Bellstate}.

\begin{figure}
    \centering
    \begin{quantikz}
        \lstick{$\ket{0}$} & \gate{H} & \ctrl{1} & \qw\rstick[wires=2]{$\ket{\Phi^+}$} \\
        \lstick{$\ket{0}$}& \qw & \targ{} & \qw
    \end{quantikz}
    \caption{Preparation of a Bell state from $\ket{0}$ using a Hadamard and CNOT.}
    \label{Bellstate}
\end{figure}
The output Bell state, cannot be written in product form. That is, the state cannot be written as,
\begin{align*}
    |{\bm\Phi}^+\rangle = & \left[ \alpha_0 |0\rangle + \beta_0|1\rangle\right] \otimes \left[\alpha_1 |0\rangle + \beta_1|1\rangle\right] \\
    =                     & \alpha_0\beta_0 |00\rangle + \alpha_0\beta_1|01\rangle + \alpha_1\beta_0|10\rangle + \alpha_1\beta_1|11\rangle
\end{align*}
since the $\alpha_0$ or $\beta_1$ must be zero in order to ensure the $|01\rangle$, $|10\rangle$ vanish.
However, this would make the coefficients of the $|00\rangle$
or $|11\rangle$ terms zero, breaking the equality. Thus, $|{\bm\Phi}^+\rangle$ cannot be written in
product form and is said to be entangled. This defines a general condition for a arbitrary state to be entangled \cite{nielsen_chuang_2010}.
