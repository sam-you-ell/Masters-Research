\section{INTRODUCTION}
Understanding the evolution of quantum many-body systems presents challenging problems and has provided insightful results into diverse areas.
Studying the dynamics of such systems is fundamentally a computational challenge due to the exponential growth of the Hilbert space with 
the number of qubits. However, quantum dynamics can be efficiently simulated in particular quantum circuits which
provide minimal models for a wide range of complex phenomena. Some examples of such quantum circuits are 
Clifford circuits and non-interacting fermi circuits, which will be of principal interest in this review. These 
circuits will form the playground in which to study the phenomena known as \textit{Quantum Scrambling}.
Quantum Scrambling describes the process in local information encoded by a simple product operator becomes 'scrambled' by a 
unitary time evolution amongst the large number of degrees of freedom in the system, such that the operator becomes a highly complicated sum 
of product operators. 
