\section{INTRODUCTION}
Understanding the evolution of quantum many-body systems presents challenging problems and has
provided insightful results in condensed matter physics and quantum information \cite{Polkovnikov_2011}.
Studying the dynamics of such systems is fundamentally
a computational challenge due to the exponential growth of the Hilbert space with the number of
qubits. However, quantum dynamics can be efficiently simulated in atypical quantum circuit models,
which will be analysed in hopes of understanding the spreading and scrambling of encoded local information,
a process known as quantum scrambling. More precisely, quantum scrambling describes the
process in which local information encoded by a simple product operator, becomes 'scrambled' by a
unitary time evolution amongst the large number of degrees of freedom in the system, such that, the
operator becomes a highly complicated sum of product operators. In recent years, the study of this scrambling process has yielded insightful
results and new perspectives. These include new insights into black hole information and the
AdS/CFT correspondence \cite{Calabrese_2009,Jerusalem,Jensen_2016,https://doi.org/10.48550/arxiv.1802.01198,Sekino_2008,ShenkerBlackHolesButterfly2014, Nozaki_2014, Calabrese_2005},
quantum chaos, \cite{Maldacena_2016, https://doi.org/10.48550/arxiv.1412.6087} and hydrodynamics in many-body systems \cite{Khemani_2018, PhysRevX.8.021013, PhysRevX.8.031058,Grozdanov_2018}.

In order to investigate this phenomenon, we turn to simulable circuit models  to gain a deeper understanding in hopes of applying any findings
to generic cases of unitary
evolution in many-body systems. As previously mentioned, some families of circuits are able to be efficiently
simulated with polynomial effort on a classical computer. Two well-studied circuit models take the principal interest of
this review, namely Clifford circuits and non-interacting fermi circuits. Clifford circuits and the Clifford
group have played a key role in the study fault-tolerant quantum computing and, more recently, many-body physics via
random Clifford unitaries \cite{PhysRevB.98.205136, https://doi.org/10.48550/arxiv.2110.02988}.
This is due to their efficient simulability on classical computers, with no constraints on the amount of entanglement,
allowing for the dynamics of large numbers of qubits to be studied via stabilizer states \cite{https://doi.org/10.48550/arxiv.2210.10129}.

Valiant \cite{Valiant2001QuantumCT} presented a new class of quantum circuits that can be simulated in
polynomial time. This class of circuits was later identified and mapped onto a non-interacting fermion system, where the Hamiltonian
describing the dynamics is quadratic in creation and annihilation operators. This particular fermion
system was shown to be classically simulable by DiVincenzo and Terhal \cite{Terhal2001} as an extension of Valiant's findings and
hence forms a good candidate to study scrambling phenomena.

With the use of recently developed techniques, the key objective of this project is to examine local
operator spreading and scrambling in models where the dynamics are analytically and numerically tractable. This is not
possible in the generic unitary evolution of many-body systems due to an exponential growth of
the Hilbert space with system size. By simulating the dynamics of scrambling in simulable circuit models, a picture of unitary evolution
can be constructed, in hopes of gaining an understanding of dynamics of generic unitary evolution
whilst examining the differences between the two cases.



