\section{Operator Scrambling}
\subsection{Operator Spreading}

As previously mentioned, quantum scrambling is the process in which local information
encoded by a simple product operator is rapidly spread over a large number of degrees of freedom via a unitary time evolution.
The information becomes highly non-local and the initial simple product operator becomes a complicated sum of product
operators. To give picture of operator spreading, consider a local operator,
$O_j$, which has support on either one site or a finite region of sites, so that it acts like the Identity on
sites without support. The simplest local operator would be a Pauli operator acting on a single site, $j$, e.g
$O_j \equiv X_j$. Then the evolution of
$O_j$, is given, in the heisenberg picture, by $O_j(t) = U^{\dagger}(t)O_jU(t)$, where $U(t)$ is some unitary time evolution
operator. Following this evolution, the local operator with minimal support, $O_j$ has evolved to $O_j(t)$ with support over a
large region of sites \cite{Khemani_2018}. Operators that grow in this way, will spread ballistically \cite{Roberts_2015, Lieb:1972wy, Schuster_2022} and are often
characterised by the out-of-time ordered correlator (OTOC) \cite{Xu2022} and the square-commutator\cite{Blake_2018}. This also gives an
intuitive picture of operator spreading, with the squared commutator defined as
\begin{equation}
    C(t) = \langle [O(t), W_i][O(t), W_i]^{\dagger}\rangle
\end{equation}
where $W_i$ is a static local operator at site $i$ \cite{https://doi.org/10.48550/arxiv.1804.08655}. Then at $t=0$, $O(t)$
acts on a single site or a finite region of sites, such that it commutes with the static operator, $W_i$ and $C(t) = 0$.
Once the operator spreads, and becomes more non-local, the commutator increases as it's support overlaps with $V_i$.

\subsection{Entanglement Growth in Operator Space}
Under a generic unitary evolution, we expect an initially simple product operator, for example, a Pauli string, $O = P_1, \dots, P_n$
to evolve into a linear superposition of Pauli strings \cite{Roberts_2018, Nahum_2017}. This entanglement growth corresponds to
an exponential increase in the possible number of Pauli strings, requiring the use of stabilizer states from Clifford circuits.
However, since Clifford unitaries only map products of Pauli operators to products of Pauli operators, a Pauli string
will remain unentangled \cite{Nahum_2018}. Despite this, the scrambling behaviour can be recovered, by the construction
of \textit{super-Clifford operators}, whilst remaining classically simulable as shown by Blake and Linden \cite{Blake2020}.

\subsubsection{Blake and Linden's Construction}
Blake and Linden introduce a family of circuits that exhibit scrambling on the space spanned by Pauli operators.
They present a gate-set of 'super-Clifford operators' that generate a near-maximal amount of operator entanglement
within the Pauli operator space, called super-Clifford circuits, when under time-evolution. These super-Clifford operators
remain classically simulable, via an extension of stabilisers to operator space, called 'super-stabilisers'.
Super-Clifford operators 'act' on Pauli operators, via conjugation in the Heisenberg picture.
The first super-Clifford operator in the Gate set is denoted as ${\bf Z.H}$, which is identified with a on Pauli operators
conjugation by a Phase gate, $T$;
\begin{align}\label{phasegate}
    T^{\dagger} X T = \frac{X - Y}{\sqrt{2}} &  & T^{\dagger} Y T = \frac{X + Y}{\sqrt{2}}
\end{align}
Following this, the operators, $X$ and $Y$ can be changed to a state-like representation, with
$X$ denoted as $[{\mathbf 0}\rangle$ and $Y$ denoted as $[{\mathbf 1}\rangle$. Then the action of $\bf Z.H$ 
can be written as,
\begin{align}
    {\bf Z.H}[{\bf 0}\rangle = \frac{[{\bf 0}\rangle - [{\bf 1}\rangle}{\sqrt{2}} \\
    {\bf Z.H}[{\bf 1}\rangle = \frac{[{\bf 0}\rangle + [{\bf 1}\rangle}{\sqrt{2}} 
\end{align}
The second gate in the set of super-Clifford operators, is the {\bf SWAP} gate,
\begin{equation}
    \text{\bf SWAP} [{\bf 01}\rangle = [{\bf 10}\rangle
\end{equation}
formed from the regular 2-qubit SWAP gate, that swaps two nearest neighbour qubits. It conjugates
Pauli operators in the following way:
\begin{equation}
    \text{SWAP}^{\dagger} X_1Y_2 \text{SWAP} = Y_1X_2
\end{equation}
The third gate, denoted $\bf C3$, acts as a combination of controlled-$\text{\bf Y}$ super-operators,
\begin{align}
    {\bf C3} [{\bf 000}\rangle=  {\bf CY}_{12}{\bf CY}_{13} [{\bf000}\rangle = [{\bf 000}\rangle \\
    {\bf C3} [{\bf 100}\rangle=  {\bf CY}_{12}{\bf CY}_{13} [{\bf100}\rangle = -[{\bf 111}\rangle
\end{align}
These three super-operators form the gate set $\{ {\bf SWAP}, {\bf Z.H}, {\bf C3}\}$, which generates entanglement
through unitary evolution in operator space, as this gate set maps Pauli strings to a linear superposition of Pauli strings.
Even though the a simple string of Pauli operators can evolve into a sum of potentially exponential operator strings,
the dynamics can be computed classically by extending the formalism of stabilizer states to operator space.
Following this, Blake and Linden showed that the gate set was capable of generating near-maximal amounts
of entanglement (slightly less than the Page value \cite{Page_1993}) among Pauli strings on a chain of 120 qubits,
quantified by the von Neumann entropy. This was only shown for operators with global support, and therefore
shows no notion of operator spreading or entanglement growth in local operators.

\subsubsection{Scrambling in Free-Fermi Circuits}
As previously mentioned, a spin 1/2 or qubit, system can be mapped onto a fermionic system via a Jordan Wigner
transformation, with the introduction of Majorana fermion operators. The evolution of Majorana fermion Operators
has been shown to exhibit operator entanglement growth by Prosen and Pi\v{z}orn \cite{Prosen_2007} by mapping an Ising spin-1/2
system onto a Majorana fermion representation. Further efforts from now on will be directed towards simulating
non-interacting fermion systems and computing the entanglement entropy so that the dynamics of local operator spreading
and entanglement growth can be fully investigated.

\section{Further Direction}
Further research will be directed towards free-fermion systems and how to simulate them. This will be carried out
by writing a python program to simulate random unitary circuits and computing the entanglement entropy. We will first
reproduce Blake and Linden's results for the $N = 120$ qubits case of super-Clifford evolution, and then go on to
simulate free-fermion circuits to confirm a non-trivial operator scrambling. Following this, the idea of combining
Clifford and non-interacting fermion circuits will be explored to investigate if any amplification of the
scrambling occurs \cite{Jozsa2008}, along with relaxing the condition of nearest-neighbour interactions, by using
long-ranged gates.



