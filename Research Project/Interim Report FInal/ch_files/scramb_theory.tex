\section{Theory}
\subsection{Quantum Information and Circuits}
\vspace{-0.15in}
Quantum Information and Quantum Computation is built upon the concept of quantum bits (qubits for short),
represented as a computational basis state of either $|0\rangle$ or $|1\rangle$. With this, the
quantum state of a system can be completely specified by writing the state as a linear combination,
$|\psi\rangle = \alpha |0\rangle + \beta |1\rangle$, where $\alpha, \beta$ are complex numbers.
This is easily extended to multi-party quantum systems of $n$ qubits. Then we may write computational
basis states as strings of qubits using the tensor product;
\[|x_1\rangle \otimes |x_2\rangle \otimes ... \otimes |x_n\rangle \equiv |x_1 x_2... x_n \rangle. \]

The dynamics of such systems can be represented via quantum circuits, made up of quantum logic gates acting
upon the qubits of a system. Analogous to a classical computer which comprised of logic gates that act upon
bit-strings of information. In contrast however, quantum logic gates are formed of linear operators, often
written in matrix form. An important set of such gates is the Pauli gates, equivalent to the set of Pauli
matrices, $P \equiv \{X, Y, Z\}$ for which $X, Y \text{ and } Z$ are defined as;
\begin{align*}
    X = \begin{bmatrix}
            0 & 1 \\
            1 & 0
        \end{bmatrix},
     &  &
    Y = \begin{bmatrix}
            0  & i \\
            -i & 0
        \end{bmatrix}
     &  &
    Z = \begin{bmatrix}
            1 & 0  \\
            0 & -1
        \end{bmatrix}
\end{align*}
These gates are all one-qubit gates, as they only act upon a single qubit. Another important one-qubit gate is
known as the Hadamard, $H$, which maps computational basis states to a superposition of computational basis states,
written explicitly in it's action;
\begin{align*}
    H |0\rangle = \frac{|0\rangle + |1\rangle}{\sqrt{2}}, \\
    H |1\rangle = \frac{|0\rangle - |1\rangle}{\sqrt{2}},
\end{align*}
or in matrix form;
\begin{equation*}
    H = \frac{1}{\sqrt{2}} \begin{bmatrix}
        1 & 1  \\
        1 & -1
    \end{bmatrix}
\end{equation*}







%PAULI GROUP DISCUSSION 
% The Pauli group on 1-qubit consists of the Pauli matrices and the Identity matrix under multiplicative factors;
% \begin{equation*}
%     G_1 \equiv \{ \pm I, \pm iI, \pm X, \pm iX \pm Y, \pm iY, \pm Z, \pm iZ\},
% \end{equation*}
% Where $\{X, Y, Z\}$ is the set of Pauli matrices, written in matrix form as;

% Extending to $n$ qubits involves taking tensor products of the matrices, such that $G_n$ consists of all
% $n$-fold tensor products of Pauli matrices, with the additional multiplicative factors to close the group.
% The Pauli group is important in quantum computation models as it allows us to describe many more quantum states
% using a construction known as \textit{Stabilizer Formalism} which will be visited when discussing the 
% Clifford group and it's corresponding .





\subsection{Operator Scrambling and Spreading}