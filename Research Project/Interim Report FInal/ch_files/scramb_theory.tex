\section{Background Theory}
\subsection{Quantum Information and Circuits}
\vspace{-0.15in}
Quantum Information is built upon the concept of quantum bits (qubits for short), represented as a linear
combination of states in the standard basis,
$|\psi\rangle = \alpha |0\rangle + \beta |1\rangle$, where $\alpha, \beta$ are complex probability amplitudes
and the vectors, $|0\rangle$, $|1\rangle$ are the computational basis states that form the standard basis.

This is easily extended to multi-party or composite systems of $n$ qubits. The total Hilbert space, $\cal{H}$ of
such a system is defined as the tensor product of $n$ subsystem Hilbert Spaces,
\begin{equation}
    {\cal{H}} = \otimes_n {\cal{H}}_n = {\cal{H}}_{n-1} \otimes {\cal{H}}_{n-2} \otimes \dots \otimes {\cal{H}}_0
\end{equation}
Then we may write the computational basis states of this system, as strings of qubits, using the tensor product \cite{nielsen_chuang_2010};
\[|x_1\rangle \otimes |x_2\rangle \otimes ... \otimes |x_n\rangle \equiv |x_1 x_2... x_n \rangle. \]

The evolution and dynamics of such systems can be represented via quantum circuits, made up of quantum logic gates acting
upon the qubits of a system. Analogous to a classical computer which is comprised of logic gates that act upon
bit-strings of information. In contrast, quantum logic gates are linear operators acting
on qubits, often represented in matrix form. This allows us to decompose a complicated unitary evolution into a
sequence of linear transformations. A common practice is to create
diagrams of such evolutions, with each quantum gate having their own symbol, analogous to circuit diagrams in classical
computation, allowing the creation of complicated quantum circuitry that can be directly mapped to a sequence of linear
operators acting on a one or more qubits. Some example gate symbols can be seen in Fig. \ref{Paulis}.


\begin{figure}[h]
    \centering
    \begin{subfloat}[pauliX]{
        \centering
        \begin{quantikz}
            &  \gate{X} 
                &  \qw
        \end{quantikz}
    }
    \end{subfloat}
    \hspace{10pt} 
    \begin{subfloat}[pauliY]{
        \centering
        \begin{quantikz}
            & \gate{Y}
                & \qw
        \end{quantikz}
    }
    \end{subfloat}
    \hspace{10pt} 
    \begin{subfloat}[pauliZ]{
        \centering
        \begin{quantikz}
            & \gate{Z}
                & \qw
        \end{quantikz}
    }
    \end{subfloat}
\end{figure}

The gates shown in Fig. \ref{Paulis} are known as the Pauli operators, equivalent
to the set of Pauli matrices, $P \equiv \{X, Y, Z\}$ for which $X, Y \text{ and } Z$ are defined in their matrix representation as;
\begin{align}
    \label{PauliMatrices}
    X = \begin{bmatrix}
            0 & 1 \\
            1 & 0
        \end{bmatrix},
     &  &
    Y = \begin{bmatrix}
            0  & i \\
            -i & 0
        \end{bmatrix}
     &  &
    Z = \begin{bmatrix}
            1 & 0  \\
            0 & -1
        \end{bmatrix}
\end{align}

These gates are all one-qubit gates, as they only act upon a single qubit.
The simplest gate is that of the Identity operator (or matrix), $I$, from which, an algebra
can be formed with the Pauli matrices. Satisfying the following relations:
\begin{align}
    XY = iZ  &  & YZ = iX  &  & ZX = iY  \\
    YX = -iZ &  & ZY = -iX &  & XZ = -iY
\end{align}
\begin{align}
    X^2 = Y^2 = Z^2 = I
\end{align}

Notably, the set of Pauli matrices and the identity form
the Pauli group, ${\cal P}_n$, defined as the $4^n$ $n$-qubit tensor products of the Pauli matrices (\ref{PauliMatrices}) and the
Identity matrix, $I$, with multiplicative factors, $\pm 1$ and $\pm i$ to ensure a legitimate group is formed.
For clarity, consider the Pauli group on 1-qubit, ${\cal P}_1$;
\begin{equation}
    {\cal P}_1 \equiv \{ \pm I, \pm iI, \pm X, \pm iX \pm Y, \pm iY, \pm Z, \pm iZ\},
\end{equation}
From this, another group of interest can be defined, namely the Clifford group, ${\cal C}_n$, defined as the
set of operators (also called unitaries) that normalise the Pauli group. The elements of this group are the
Hadamard, Controlled-Not and Phase operators.

The Hadamard, $H$, maps computational basis states to a superposition of computational basis states,
written explicitly in it's action;
\begin{align*}
    H |0\rangle = \frac{|0\rangle + |1\rangle}{\sqrt{2}}, \\
    H |1\rangle = \frac{|0\rangle - |1\rangle}{\sqrt{2}},
\end{align*}
or in matrix form;
\begin{equation*}
    H = \frac{1}{\sqrt{2}} \begin{bmatrix}
        1 & 1  \\
        1 & -1
    \end{bmatrix}
\end{equation*}
Controlled-Not ($CNOT$) is a two-qubit gate. One qubit acts as a control for an operation to be perfomed
on the other qubit. It's matrix representation is
\begin{align}
    \label{CNOT}
    CNOT =
    \renewcommand{\arraystretch}{0.75}
    \begin{bmatrix}
        1 & 0 & 0 & 0 \\
        0 & 1 & 0 & 0 \\
        0 & 0 & 0 & 1 \\
        0 & 0 & 1 & 0
    \end{bmatrix}
\end{align}
The Phase operator, denoted $S$ is defined as,
\begin{align*}
    S =
    \begin{bmatrix}
        1 & 0                  \\
        0 & e^{i\frac{\pi}{2}}
    \end{bmatrix}
\end{align*}

The CNOT operator is often used to create entanglement in circuits to
generate states that cannot be written in product form. For example, if we prepare a Bell state
$|{\bm\Phi}^{+}\rangle_ = \frac{|00\rangle + |11\rangle}{\sqrt{2}}$. We cannot write this as
\begin{align*}
    |{\bm\Phi}^+\rangle = & \left[ \alpha_0 |0\rangle + \beta_0|1\rangle\right] \otimes \left[\alpha_1 |0\rangle + \beta_1|1\rangle\right] \\
    =                     & \alpha_0\beta_0 |00\rangle + \alpha_0\beta_1|01\rangle + \alpha_1\beta_0|10\rangle + \alpha_1\beta_1|11\rangle
\end{align*}
since the $\alpha_0$ or $\beta_1$ must be zero in order to ensure the $|01\rangle$, $|10\rangle$ vanish.
However, this would make the coefficients of the $|00\rangle$
or $|11\rangle$ terms zero, breaking the equality. Thus, $|{\bm\Phi}^+\rangle$ cannot be written in
product form and is said to be entangled. This defines a general condition for a arbitrary state to be entangled.
\subsection{Clifford Circuits and Stabilizer Formalism}

Quantum circuits comprised of only quantum gates from the Clifford group are known as \textit{Clifford Circuits}.
This family of circuits is of considerable interest as they are said to be classically simulable, that is,
circuits of this construction can be efficiently simulated on a classical computer with polynomial
effort via the Gottesman-Knill theorem \cite{knillGottesman}. For this reason, Clifford gates do not form
a set of universal quantum gates, meaning a universal quantum computer cannot be constructed using a set
of Clifford gates. A set of universal quantum gates allows for any unitary operation to be approximated to
arbitrary accuracy by a quantum circuit, constructed using only the original set of gates.

\textit{Stabilizer Formalism}. An arbitrary, pure quantum state, $|\psi\rangle$ is \textit{stabilized} by
a unitary operator \footnote{I will call Unitary operators, Unitaries to be in line with literature and for the sake of word count},
$U$ if $|\psi\rangle$  is an eigenvector of $U$, with eigenvalue 1, satisfying:
\begin{equation}
    U|\psi\rangle = |\psi\rangle
\end{equation}
The key idea here is that the quantum state can be described in terms of the unitaries that stabilize it, instead of the state itself.
This is more appealing, as mathematical techniques from group theory are able to be exploited to more
compactly describe the dynamics of unitary evolutions.

\subsection{Free-Fermion Circuits}











