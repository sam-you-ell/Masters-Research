\documentclass[twocolumn, aps, nobalancelastpage]{revtex4-2}

% \usepackage[utf8]{inputenc}
% \usepackage[T1]{fontenc}
%\usepackage{hyperref}

\usepackage{graphicx}
\usepackage{tikz}
\usetikzlibrary{quantikz}
\usepackage{amsmath, amsthm, amssymb, mathtools}
\usepackage{fancyhdr}
\usepackage{geometry}
\usepackage{bm}
\usepackage{subfig}
\usepackage{multirow}
\usepackage{array}
\usepackage{hyperref}
\newtheorem{theorem}{Theorem}



                      

\fancyhf{}
\fancyheadoffset{-0.001\textwidth}
\pagestyle{fancy}
\setlength{\textwidth}{7in}
\setlength{\evensidemargin}{-0.2in}
\setlength{\oddsidemargin}{-0.2in}
\setlength{\headheight}{14pt}
%\setlength{\headwidth}{6in}
\setlength{\topmargin}{-0.5in}
\setlength{\textheight}{8.4in}
% \setlength{\baselineskip}{10pt}

\begin{document}

\title{A Review on Quantum Scrambling within Classically Simulable Circuit Models}

\author{Samuel A. Hopkins$^{1*}$}
\affiliation{$^1$H. H. Wills Physics Laboratory, University of Bristol, Bristol, BS8 1TL, UK}
\date{\today}
\thanks{Email: cn19407@bristol.ac.uk\\
Supervisor: Stephen R. Clark\\
Word Count: 2904
}



% \begin{abstract}
    Abstract Goes Here...
\end{abstract}
\maketitle

\tableofcontents

\fancyhead[L]{\textit{Samuel A. Hopkins}}
\fancyhead[R]{Quantum Scrambling Review}
\section{INTRODUCTION}
Understanding the evolution of quantum many-body systems presents challenging problems and has
provided insightful results in condensed matter physics and quantum information \cite{Polkovnikov_2011}.
Studying the dynamics of such systems is fundamentally
a computational challenge due to the exponential growth of the Hilbert space with the number of
qubits. However, quantum dynamics can be efficiently simulated in atypical quantum circuit models,
which will be analysed in hopes of understanding the spreading and scrambling of encoded local information,
a process known as quantum scrambling. More precisely, quantum scrambling describes the
process in which local information encoded by a simple product operator, becomes 'scrambled' by a
unitary time evolution amongst the large number of degrees of freedom in the system, such that, the
operator becomes a highly complicated sum of product operators. In recent years, the study of this scrambling process has yielded insightful
results and new perspectives. These include new insights into black hole information and the
AdS/CFT correspondence \cite{Calabrese_2009,Jerusalem,Jensen_2016,https://doi.org/10.48550/arxiv.1802.01198,Sekino_2008,ShenkerBlackHolesButterfly2014, Nozaki_2014, Calabrese_2005},
quantum chaos, \cite{Maldacena_2016, https://doi.org/10.48550/arxiv.1412.6087} and hydrodynamics in many-body systems \cite{Khemani_2018, PhysRevX.8.021013, PhysRevX.8.031058,Grozdanov_2018}.

In order to investigate this phenomenon, we turn to simulable circuit models  to gain a deeper understandingin hopes of applying any findings
to generic cases of unitary
evolution in many-body systems. As previously mentioned, some families of circuits are able to be efficiently
simulated with polynomial effort on a classical computer. Two well-studied circuit models take the principal interest of
this review, namely Clifford circuits and non-interacting fermi circuits. Clifford circuits and the Clifford
group have played a key role in the study fault-tolerant quantum computing and, more recently, many-body physics via
random Clifford unitaries \cite{PhysRevB.98.205136, https://doi.org/10.48550/arxiv.2110.02988}.
This is due to their efficient simulability on classical computers, with no constraints on the amount of entanglement,
allowing for the dynamics of large numbers of qubits to be studied via stabilizer states \cite{https://doi.org/10.48550/arxiv.2210.10129}.

Valiant \cite{Valiant2001QuantumCT} presented a new class of quantum circuits that can be simulated in
polynomial time. This class of circuits was later identified and mapped onto a non-interacting fermion system, where the Hamiltonian
describing the dynamics is quadratic in creation and annihilation operators. This particular fermion
system was shown to be classically simulable by DiVincenzo and Terhal \cite{Terhal2001} as an extension of Valiant's findings and 
hence forms a good candidate to study scrambling phenomena.

With the use of recently developed techniques, the key objective of this project is to examine local
operator spreading and scrambling in models where the dynamics are analytically and numerically tractable. This is not
possible in the generic unitary evolution of many-body systems due to an exponential growth of
the Hilbert space with system size. By simulating the dynamics of scrambling in simulable circuit models, a picture of unitary evolution
can be constructed, in hopes of gaining an understanding of dynamics of generic unitary evolution
whilst examining the differences between the two cases.





\section{Background Theory}
\subsection{Quantum Information and Computing}
\vspace{-0.15in}
Quantum Information and Computation is built upon the concept of quantum bits (qubits for short), represented as a linear
combination of states in the standard basis,
$|\psi\rangle = \alpha |0\rangle + \beta |1\rangle$, where $\alpha, \beta$ are complex probability amplitudes
and the vectors, $|0\rangle$, $|1\rangle$ are the computational basis states that form the standard basis.

This is extended to multi-party or composite systems of $n$ qubits via the tensor product. The total Hilbert space, $\cal{H}$ of
a many-body system is defined as the tensor product of $n$ subsystem Hilbert Spaces,
\begin{equation}
    {\cal{H}} = \otimes_n {\cal{H}}_n = {\cal{H}}_{1} \otimes {\cal{H}}_{2} \otimes \dots \otimes {\cal{H}}_n.
\end{equation}
The computational basis states of this system, are tensor products of qubit states, often written as a string \cite{schumacher_westmoreland_2010},
\[|x_1\rangle \otimes |x_2\rangle \otimes ... \otimes |x_n\rangle \equiv |x_1 x_2... x_n \rangle. \]

The evolution and dynamics of many-body systems can be represented via quantum circuits, constructed from a set of quantum logic gates acting
upon the qubits of a system. Analogous to a classical computer which is comprised of logic gates that act upon
bit-strings of information. In contrast, quantum logic gates are linear operators acting
on qubits, often represented in matrix form. This allows for the decomposition of a unitary evolution into a
sequence of linear transformations. A common practice is to create
diagrams of such evolutions, with each quantum gate having their own symbol, analogous to circuit diagrams in classical
computation, allowing the creation of complicated quantum circuitry that can be directly mapped to a sequence of linear
operators acting on a one or more qubits. Some example gate symbols can be seen in Fig. \ref{Paulis}.


\begin{figure}[h]
    \centering
    \begin{subfloat}[pauliX]{
        \centering
        \begin{quantikz}
            &  \gate{X} 
                &  \qw
        \end{quantikz}
    }
    \end{subfloat}
    \hspace{10pt} 
    \begin{subfloat}[pauliY]{
        \centering
        \begin{quantikz}
            & \gate{Y}
                & \qw
        \end{quantikz}
    }
    \end{subfloat}
    \hspace{10pt} 
    \begin{subfloat}[pauliZ]{
        \centering
        \begin{quantikz}
            & \gate{Z}
                & \qw
        \end{quantikz}
    }
    \end{subfloat}
\end{figure}

The gates shown in Fig. \ref{Paulis} are known as the Pauli operators, equivalent
to the set of Pauli matrices, $P \equiv \{X, Y, Z\}$ for which $X, Y \text{ and } Z$ are defined in their matrix representation as
\begin{align}
    \label{PauliMatrices}
    X = \begin{bmatrix}
            0 & 1 \\
            1 & 0
        \end{bmatrix},
     &  &
    Y = \begin{bmatrix}
            0  & i \\
            -i & 0
        \end{bmatrix},
     &  &
    Z = \begin{bmatrix}
            1 & 0  \\
            0 & -1
        \end{bmatrix},
\end{align}

These gates are all one-qubit gates, as they only act upon a single qubit.
Together with the Identity operator, $I$, the Pauli matrices form an algebra .
Satisfying the following relations
\begin{align}
    XY = iZ,  &  & YZ = iX,  &  & ZX = iY,  \\
    YX = -iZ, &  & ZY = -iX, &  & XZ = -iY,
\end{align}
\begin{align}
    X^2 = Y^2 = Z^2 = I.
\end{align}

Notably, the set of Pauli matrices and the identity form
the Pauli group, ${\cal P}_n$, defined as the $4^n$ $n$-qubit tensor products of the Pauli matrices (\ref{PauliMatrices}) and the
Identity matrix, $I$, with multiplicative factors, $\pm 1$ and $\pm i$ to ensure a legitimate group is formed.
For clarity, consider the Pauli group on 1-qubit, ${\cal P}_1$;
\begin{equation}
    {\cal P}_1 \equiv \{ \pm I, \pm iI, \pm X, \pm iX \pm Y, \pm iY, \pm Z, \pm iZ\},
\end{equation}
From this, another group of interest can be defined, namely the Clifford group, ${\cal C}_n$, defined as a
subset of unitary operators that normalise the Pauli group \cite{orthogonalcodes} *INSERT CLIFFORD GROUP DEFINITION*.
The elements of this group are the
Hadamard, Controlled-Not and Phase operators.

The Hadamard, $H$, maps computational basis states to a superposition of computational basis states,
written explicitly in it's action;
\begin{align*}
    H |0\rangle = \frac{|0\rangle + |1\rangle}{\sqrt{2}}, \\
    H |1\rangle = \frac{|0\rangle - |1\rangle}{\sqrt{2}},
\end{align*}
or in matrix form;
\begin{equation*}
    H = \frac{1}{\sqrt{2}} \begin{bmatrix}
        1 & 1  \\
        1 & -1
    \end{bmatrix}
\end{equation*}
Controlled-Not ($CNOT$) is a two-qubit gate. One qubit acts as a control for an operation to be perfomed
on the other qubit.
It's matrix representation is
\begin{align}
    \label{CNOT}
    CNOT =
    \renewcommand{\arraystretch}{0.75}
    \begin{bmatrix}
        1 & 0 & 0 & 0 \\
        0 & 1 & 0 & 0 \\
        0 & 0 & 0 & 1 \\
        0 & 0 & 1 & 0
    \end{bmatrix}.
\end{align}
The Phase operator, denoted $S$ is defined as,
\begin{align*}
    S =
    \begin{bmatrix}
        1 & 0                  \\
        0 & e^{i\frac{\pi}{2}}
    \end{bmatrix}.
\end{align*}

The CNOT operator is often used to an generate entangled state. One such state is the maximally entangled 2-qubit state,
called a Bell state, $|{\bm\Phi}^{+}\rangle_ = (|00\rangle + |11\rangle)/\sqrt{2}$. This is prepared
from a $|00\rangle$ state, by applying a Hadamard to the first qubit, and subsequently a Controlled-Not gate
\begin{align}
    H \otimes I |00\rangle = \left (\frac{|0\rangle + |1\rangle }{\sqrt{2}}\right )|0\rangle \\
    CNOT \left (\frac{|0\rangle + |1\rangle }{\sqrt{2}}\right )|0\rangle = \frac{|00\rangle + |11\rangle}{\sqrt{2}}
\end{align}
The corresponding circuit representation of this preparation is given in Fig. \ref{Bellstate}.

\begin{figure}
    \centering
    \begin{quantikz}
        \lstick{$\ket{0}$} & \gate{H} & \ctrl{1} & \qw\rstick[wires=2]{$\ket{\Phi^+}$} \\
        \lstick{$\ket{0}$}& \qw & \targ{} & \qw
    \end{quantikz}
    \caption{Preparation of a Bell state from $\ket{0}$ using a Hadamard and CNOT.}
    \label{Bellstate}
\end{figure}
The output Bell state, cannot be written in product form. That is, the state cannot be written as,
\begin{align*}
    |{\bm\Phi}^+\rangle = & \left[ \alpha_0 |0\rangle + \beta_0|1\rangle\right] \otimes \left[\alpha_1 |0\rangle + \beta_1|1\rangle\right] \\
    =                     & \alpha_0\beta_0 |00\rangle + \alpha_0\beta_1|01\rangle + \alpha_1\beta_0|10\rangle + \alpha_1\beta_1|11\rangle
\end{align*}
since the $\alpha_0$ or $\beta_1$ must be zero in order to ensure the $|01\rangle$, $|10\rangle$ vanish.
However, this would make the coefficients of the $|00\rangle$
or $|11\rangle$ terms zero, breaking the equality. Thus, $|{\bm\Phi}^+\rangle$ cannot be written in
product form and is said to be entangled. This defines a general condition for a arbitrary state to be entangled \cite{nielsen_chuang_2010}.

\section{Classically Simulable Quantum Circuits}
%Maybe some introduction, short sentence on what classical simulability is.
\subsection{Clifford Circuits and Stabilizer Formalism}

Quantum circuits comprised of only quantum gates from the Clifford group are known as \textit{Clifford Circuits}.
This family of circuits are of considerable interest due to their classical simulability, that is,
circuits of this construction can be efficiently simulated on a classical computer with polynomial
effort via the Gottesman-Knill theorem. For this reason, Clifford gates do not form
a set of universal quantum gates, meaning a universal quantum computer cannot be constructed using a set
of Clifford gates. A set of universal quantum gates allows for any unitary operation to be approximated to
arbitrary accuracy by a quantum circuit, constructed using only the original set of gates. Clifford
circuits are also known as Stabilizer circuits.

\textit{Stabilizer Formalism}. An arbitrary, pure quantum state, $|\psi\rangle$ is \textit{stabilized} by
a unitary operator, $S$ if $|\psi\rangle$  is an eigenvector of $S$, with eigenvalue 1, satisfying:
\begin{equation}
    S|\psi\rangle = |\psi\rangle
\end{equation}
The key idea here is that the quantum state can be described in terms of the unitaries that stabilize it, instead of the state itself.
This is due to $|\psi\rangle$ being the unique state that is stabilized by $S$. \cite{PhysRevA.70.052328}
This can be seen from considering the Pauli matrices, and the unique states they stabilize. In the one-qubit case these are
the $+1$ eigenstates of the pauli matrices (omitting normalisation factors);
\begin{align}
     & X(|0\rangle + |1\rangle) = |0\rangle + |1\rangle   \\
     & Y(|0\rangle + i|1\rangle) = |0\rangle + i|1\rangle \\
     & Z|0\rangle = |0\rangle
\end{align}

If given a group or subgroup of unitaries, $\cal U$, the vector space, $V$, of $n$ qubit states is
stabilized by $\cal U$ if every element of $V$ is stable under action from every element of $\cal U$.
This description is more appealing, as we can exploit mathematical techniques from group theory to
describe quantum states and vector spaces. More precisely, the group $\cal U$ can be described using it's
generators. In general, a set of elements  $g_1, \dots, g_d$ of a group, $\cal G$, generate the group
if every element of $\cal G$ can be written as a product of elements from the set of generators,
$g_1, \dots, g_d$, such that ${\cal G} \coloneqq \langle g_1, \dots, g_d\rangle$. For example,
consider the group ${\cal U} \equiv \{ I, Z_1Z_2, Z_2Z_3, Z_1Z_3\}$. ${\cal U}$ can be compactly written
as ${\cal U} = \langle Z_1Z_2, Z_2Z_3 \rangle$, by recognising $(Z_1Z_2)^2 = I$ and
$Z_1Z_3 = (Z_1Z_2)(Z_2Z_3)$.

This allows for the description of a quantum state, and subsequently it's dynamics, in terms of the generators
of a stabilizer group. To see how the dynamics of a state are represented in terms of generators, consider
a stabilizer state under the action of an arbitrary unitary operator:
\begin{align}
    UM|\psi \rangle =  UMU^{\dagger}U|\psi\rangle
\end{align}
The state $|\psi \rangle$ is an eigenvector of $M$ if and only if, $U |\psi\rangle$ is an
eigenvector of $UMU^{\dagger}$. Thus, the application of an unitary operator transforms
$M \to UMU^{\dagger}$. Moreover, if the state $|\psi \rangle$ is stabilized by $M$, then the evolved
state $U|\psi\rangle$ will be stabilized by $UMU^{\dagger}$. If $M$ is an element of a stabilizer group
$\cal S$ such that $M_1, \dots, M_l$ generate $\cal S$, then $UM_1U^{\dagger}, \dots, UM_lU^{\dagger}$ must generate
$U{\cal S}U^{\dagger}$. This implies that to compute the dynamics of a stabilizer, only the transformation of
the generators needs to be considered \cite{fault-tolerantQC}. It is because of this,
Clifford circuits are able to be efficiently classically simulated via the Gottesman-Knill theorem.
\begin{theorem}[Gottesman-Knill Theorem \cite{knillGottesman}]

    Given an $n$ qubit state $|\psi \rangle$, the following statements are equivalent:
    \begin{itemize}
        \item $|\psi \rangle$ can be obtained from $|0 \rangle^{\otimes n}$ by CNOT, Hadamard and phase gates only.
        \item $|\psi \rangle$ is stabilized by exactly $2^n$ Pauli operators
        \item $|\psi \rangle$ can be uniquely identified by the group of Pauli operators that
              stabilize $|\psi \rangle$.
    \end{itemize}
    %possibility for discussion of bits
    % Could also extend to qudits
\end{theorem}

\subsection{Free-Fermion Circuits}
Another class of circuits that are classically simulable, are known as free-fermion circuits. These
circuits were built from a special set of 2 qubit gates, acting on nearest neighbouring qubits, first
introduced by Valiant \cite{Valiant2001QuantumCT} and later mapped onto a non-interacting fermion system
by DiVincenzo and Terhal \cite{Terhal2001}.

% INSERT TWO QUBIT GATE HERE ALONG WITH HAMILTONIANS

The fermion system is setup as $n$ qubit computational
basis states, $|x_0, \dots, x_{n-1}\rangle$ which are identified with $n$ local fermionic modes
(LFM's), each of which having an occupation number $n_j = 0,1$ at the $j$-th mode. $n_j = 0$ corresponds
to an unoccupied mode, while $n_j = 1$ corresponds to an occupied mode. The mapping of a qubit system (spin 1/2 system) onto
a fermionic system is done via a Jordan-Wigner Transformation, such that the representation and dynamics of a quantum
state are described using creation and annihilation operators from second quantization.
The creation operator, $a_j^{\dagger}$ creates a fermion at mode $j$ (if the mode is unoccupied),
whilst the annihilation operator, $a_j$ removes a fermion at site $j$ if the mode is occupied.
In the case of fermions, these operators obey canonical anti-commutation relations:
\begin{align}
    \{a_j, a_k\} \equiv \{a_j^{\dagger}, a_k^{\dagger}\} = 0 &  & \{a_j, a_k^{\dagger}\} = \delta_{jk}I
\end{align}
The Hilbert space of such a system is called a Fock space, for which the basis vectors are Fock states,
$|n_1, n_2, \dots n_l\rangle$. The elements of which, are the occupation number at each mode.
The creation and annihilation operators act on Fock states in the following way:
\begin{align*}
    a_j^{\dagger} |\mathbf{n}\rangle = (-1)^{\sum^{j-1}_{m=0}}\delta_{n_j, 1}  |{\mathbf n}'\rangle \\
    a_j |\mathbf{n}\rangle = (-1)^{\sum^{j-1}_{m=0}}\delta_{n_j, 0}  |{\mathbf n}'\rangle
\end{align*}
Where $ |{\mathbf n}\rangle = |n_1, \dots, n_j, \dots, n_l\rangle$ and $ |{\mathbf n}'\rangle = |n_1, \dots, n'_j, \dots, n_l\rangle$
includes the updated occupation number at site $j$.
Analagous to stabilizer states, an arbitrary Fock state, $|{\mathbf n}\rangle$, can be created from
a vacuum state via creation operators,
\begin{equation}
    |{\mathbf n}\rangle = (a_1^{\dagger})^{n_1} (a_2^{\dagger})^{n_2} \dots (a_{l}^{\dagger})^{n_l}|{\mathbf 0}\rangle
\end{equation}
where $|{\mathbf 0}\rangle = |0,0,\dots,0\rangle$ is a vacuum state of size $l$. Then a number operator can be defined,
$\hat{n}_j = c^{\dagger}_j c_j$, such that it's action on a Fock state returns the occupation number on site $j$.
As previously mentioned, the Pauli operators can be mapped onto a non-interacting fermion system via
the Jordan-Wigner transformation (JWT). In particular, the set of Pauli operators that act on the $j$th site,
$\{X_j, Y_j, Z_j\}$, can be expressed in terms of the creation and annihlation operators by first defining
the operators $\sigma^{\pm}_j = \frac{1}{2}(X_j \pm iY_j)$. Then in terms of the creation and annihilation
operators:
\begin{align}
    \sigma^{+}_j = e^{i\pi\sum^{j-1}_{m = 0} a^{\dagger}_m a_m} a_j^{\dagger} \\
    \sigma^{-}_j = e^{i\pi\sum^{j-1}_{m = 0} a^{\dagger}_m a_m} a_j
\end{align}
















\section{Operator Scrambling}
\subsection{Operator Spreading}

As previously mentioned, quantum scrambling is the process in which local information
encoded by a simple product operator is rapidly spread over a large number of degrees of freedom via a unitary time evolution.
The information becomes highly non-local and the initial simple product operator becomes a complicated sum of product
operators. To give picture of operator spreading, consider a local operator,
$O_j$, which has support on either one site or a finite region of sites, so that it acts like the Identity on
sites without support. The simplest local operator would be a Pauli operator acting on a single site, $j$, e.g
$O_j \equiv X_j$. Then the evolution of
$O_j$, is given, in the heisenberg picture, by $O_j(t) = U^{\dagger}(t)O_jU(t)$, where $U(t)$ is some unitary time evolution
operator. Following this evolution, the local operator with minimal support, $O_j$ has evolved to $O_j(t)$ with support over a
large region of sites \cite{Khemani_2018}. Operators that grow in this way, will spread ballistically \cite{Roberts_2015, Lieb:1972wy, Schuster_2022} and are often
characterised by the out-of-time ordered correlator (OTOC) \cite{Xu2022} and the square-commutator\cite{Blake_2018}. This also gives an
intuitive picture of operator spreading, with the squared commutator defined as
\begin{equation}
  C(t) = \langle [O(t), W_i][O(t), W_i]^{\dagger}\rangle
\end{equation}
where $W_i$ is a static local operator at site $i$ \cite{https://doi.org/10.48550/arxiv.1804.08655}. Then at $t=0$, $O(t)$
acts on a single site or a finite region of sites, such that it commutes with the static operator, $W_i$ and $C(t) = 0$.
Once the operator spreads, and becomes more non-local, the commutator increases as it's support overlaps with $V_i$.

\subsection{Entanglement Growth in Operator Space}
Under a generic unitary evolution, we expect an initially simple product operator, for example, a Pauli string, $O = P_1, \dots, P_n$
to evolve into a linear superposition of Pauli strings \cite{Roberts_2018, Nahum_2017}. This entanglement growth corresponds to
an exponential increase in the possible number of Pauli strings, requiring the use of stabilizer states from Clifford circuits *INSERT
CLASSICAL COMPLEXITY OF QUANTUM CIRCUITS*.
However, since Clifford unitaries only map products of Pauli operators to products of Pauli operators, a Pauli string
will remain unentangled \cite{Nahum_2018}. Despite this, the scrambling behaviour can be recovered, by the construction
of \textit{super-Clifford operators}, whilst remaining classically simulable as shown by Blake and Linden \cite{Blake2020}.

\subsubsection{Blake and Linden's Construction}
Blake and Linden introduce a family of circuits that exhibit scrambling on the space spanned by Pauli operators.
They present a gate-set of 'super-Clifford operators' that generate a near-maximal amount of operator entanglement
within the Pauli operator space, called super-Clifford circuits, when under time-evolution. These super-Clifford operators
remain classically simulable, via an extension of stabilisers to operator space, called 'super-stabilisers'.
Super-Clifford operators 'act' on Pauli operators, via conjugation in the Heisenberg picture.
The first super-Clifford operator in the Gate set is denoted as ${\bf Z.H}$, which is identified with a on Pauli operators
conjugation by a Phase gate, $T$;
\begin{align}\label{phasegate}
  T^{\dagger} X T = \frac{X - Y}{\sqrt{2}}, &  & T^{\dagger} Y T = \frac{X + Y}{\sqrt{2}}.
\end{align}
Following this, the operators, $X$ and $Y$ can be changed to a state-like representation, with
$X$ denoted as $[{\mathbf 0}\rangle$ and $Y$ denoted as $[{\mathbf 1}\rangle$. Then the action of $\bf Z.H$
can be written as,
\begin{align}
  {\bf Z.H}[{\bf 0}\rangle = \frac{[{\bf 0}\rangle - [{\bf 1}\rangle}{\sqrt{2}}, \\
  {\bf Z.H}[{\bf 1}\rangle = \frac{[{\bf 0}\rangle + [{\bf 1}\rangle}{\sqrt{2}}.
\end{align}
The second gate in the set of super-Clifford operators, is the {\bf SWAP} gate,
\begin{equation}
  \text{\bf SWAP} [{\bf 01}\rangle = [{\bf 10}\rangle,
\end{equation}
formed from the regular 2-qubit SWAP gate, that swaps two nearest neighbour qubits. It conjugates
Pauli operators in the following way
\begin{equation}
  \text{SWAP}^{\dagger} X_1Y_2 \text{SWAP} = Y_1X_2.
\end{equation}
The third gate, denoted $\bf C3$, acts as a combination of controlled-$\text{\bf Y}$ super-operators,
\begin{align}
  {\bf C3} [{\bf 000}\rangle=  {\bf CY}_{12}{\bf CY}_{13} [{\bf000}\rangle = [{\bf 000}\rangle, \\
  {\bf C3} [{\bf 100}\rangle=  {\bf CY}_{12}{\bf CY}_{13} [{\bf100}\rangle = -[{\bf 111}\rangle.
\end{align}
These three super-operators form the gate set $\{ {\bf SWAP}, {\bf Z.H}, {\bf C3}\}$, which generates entanglement
through unitary evolution in operator space, as this gate set maps Pauli strings to a linear superposition of Pauli strings.
Despite the fact a simple string of Pauli operators can evolve into a sum of potentially exponential operator strings,
the dynamics can be computed classically by extending the formalism of stabilizer states to operator space.
Following this, Blake and Linden showed that the gate set was capable of generating near-maximal amounts
of entanglement (slightly less than the Page value \cite{Page_1993}) among Pauli strings on a chain of 120 qubits,
quantified by the von Neumann entropy. This was only shown for operators with global support, and therefore
shows no notion of operator spreading or entanglement growth in local operators.

\subsubsection{Scrambling in Free-Fermi Circuits}
As previously mentioned, a spin 1/2 or qubit, system can be mapped onto a fermionic system via a Jordan Wigner
transformation, with the introduction of Majorana fermion operators. The evolution of Majorana fermion Operators
has been shown to exhibit operator entanglement growth by Prosen and Pi\v{z}orn \cite{Prosen_2007} by mapping an Ising spin-1/2
system onto a Majorana fermion representation. Further efforts from now on will be directed towards simulating
non-interacting fermion systems and computing the entanglement entropy so that the dynamics of local operator spreading
and entanglement growth can be fully investigated.

\section{Further Direction}
Further research will be directed towards free-fermion systems and how to simulate them. This will be carried out
by writing a python program to simulate random unitary circuits and computing the entanglement entropy. We will first
reproduce Blake and Linden's results for the $N = 120$ qubits case of super-Clifford evolution, and then go on to
simulate free-fermion circuits to confirm a non-trivial operator scrambling. Following this, the idea of combining
Clifford and non-interacting fermion circuits will be explored to investigate if any amplification of the
scrambling occurs \cite{Jozsa2008}, along with relaxing the condition of nearest-neighbour interactions, by using
long-ranged gates.







\bibliographystyle{ieeetr}
\bibliography{QuantumScrambling.bib}

\end{document}