
\subsection{Fermionic Systems}
%Indistinguishibility. 

The familiar qubit system may be mapped onto a system of identical particles (fermions), such that the 
overall many body state describing the system, is invariant under particle exchange. This is 
performed via a Jordan-Wigner transformation, which maps any local spin-model to a local fermionic model.
To gain an understanding of how this can be carried out and why it is relevant, it will be useful to introduce the core concepts and language from \textit{Second Quantization}.

\subsubsection{Second Quantization and Indistinguishable Particles}


The wave function for a system of $N$ identical particles is $\psi(x_1, x_2, \dots x_N)$, where a particle is specified by it's position vector, $\vec{x}$. For bosonic systems, the wavefunction is symmetric under particle exhchange, while fermionic systems present anti-symmetric wavefunctions under particle exchange. 

% For a specified quantum many-body state, wherein the particles are identical, if two particles are exchanged the initial many-body state will remain unchanged.
% (This is different) to classical mechanics, where if particles' position vectors are exchanged, the resulting many-body state is an entirely new configuration. 

The construction of the $n$ particle state via the extension of the single particle wavefunction, as described in (\refeq{nqubit}), 
leads to a redundancy in it's description of a many-body state and an unnecessarily large Hilbert space. 



A more efficient appraoch to describing many-body states is the formalism of second quantisation.Instead of describing states with Slater determinants, a \textit{Fock state} presents an elegant and convinient basis in which to work in. The Fock state of a many-body system is represented in an occupancy number basis, written as $|n_1, n_2, \dots, n_L\rangle$, where $n_i$ is the occupation number. For wavefunctions that are symmetric under particle exchange, i.e bosonic systems, the occupancy can take any real non-negative integer. For wavefunctions that are anti-symmetric under particle exchange (fermionic systems), the occupancy number is either $0$ or $1$ such that the state satisfies the Pauli exclusion principle.

To preserve the symmetric properties of Fock states, second quantization introduces fermionic creation and annihilation operators. The creation operator, $a^{\dagger}_{i}$ creates a particle at site $i$ if unoccupied, and the annihilation operator, $a_i$, removes a particle at site $i$ if occupied. More formally, it's action on a Fock state may be written as, 
\begin{align}
    a_j^{\dagger} |n_0, \dots, n_j, \dots, n_{m-1}\rangle = {(-1)}^{\sum^{j-1}_{s=0}n_s} (1-n_j) |n_0, \dots, n_j - 1, \dots, n_{m-1}\rangle, \\
    a_j |n_0, \dots, n_j, \dots, n_{m-1}\rangle = {(-1)}^{\sum^{j-1}_{s=0}n_s} n_j |n_0, \dots, n_j - 1, \dots, n_{m-1}\rangle.
\end{align}
Using these operators, an arbitrary Fock state, $|\psi_{F}\rangle$ may be constructed from a set of creation operators acting on the vacuum, 
\begin{equation}
    |\psi_{f}\rangle = (a_{1}^{\dagger})^{n_1} (a_{2}^{\dagger})^{n_2} \dots (a_{L}^{\dagger})^{n_{L}}|{\mathbf 0}\rangle.
\end{equation}
The fermionic creation and annihilation operators obey crucial anti-commutation relations, constructed to 
\begin{align}
    \{a_j, a_k\} \equiv \{a_j^{\dagger}, a_k^{\dagger}\} = 0, && \{a_j, a_k^{\dagger}\} = \delta_{jk}I
\end{align}
As an alternative description, the the \textit{Majorana operators} may be defined as 
$c_{2k} = a_k + a_k^{\dagger}$ and $c_{2k+1} = i(a_k^{\dagger} - a_{k})$ for $k =  1, \dots, N$ such that they satisfy $\{c_j, c_k\} = 2\delta_{jk}$. From this, the Jordan-Wigner transformation may be defined, which maps a system of $N$ spin-$\frac{1}{2}$ particles onto a system of $N$ `spinless' fermions as by defining the Majorana operators onto Pauli spin operators: 

\begin{align}
    c
\end{align}
Such a mapping allows the Hilbert space of $N$ qubits to be identified with the Hilbert space of $N$ local fermionic modes (LFM's). 
%introduce majorana operators


%possible appendix here 




