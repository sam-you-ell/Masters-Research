\newpage
\section{Fermionic Systems}
%Indistinguishibility. 

The familiar qubit system may be mapped onto a system of identical particles (fermions), such that the 
overall many body state describing the system, is invariant under particle exchange. This is 
performed via a Jordan-Wigner transformation, which maps any local spin-model to a local fermionic model.
To gain an understanding of how this can be carried out and why it is relevant, it will be useful to introduce 
the core concepts and language from \textit{Second Quantization}.

\subsection{Second Quantization and Indistinguishable Particles}

A key distinction between classical and quantum mechanics, lies in the notion of the indistinguishability of particles.
In quantum mechanics, particles are treated as identical and the exhanging of particles leaves the quantum many-body state unchanged, as opposed to classical mechanics
where permutations on the set of particles' position vectors results in a new many-body state. 

% For a specified quantum many-body state, wherein the particles are identical, if two particles are exchanged the initial many-body state will remain unchanged.
% (This is different) to classical mechanics, where if particles' position vectors are exchanged, the resulting many-body state is an entirely new configuration. 

The construction of the $n$ particle state via the extension of the single particle wavefunction, as described in (\refeq{nqubit}), 
leads to a redundancy in it's description of a many-body state and an unnecessarily large Hilbert space. Second quantization remedies this redundancy by only considering the number of particles in each state, resulting in an 
efficient formalism to describe many-body systems. 

In second quantization, the state of a many-body system is represented in an occupancy number basis, known as Fock states. For a given configuration, a Fock state may be written as $|n_1, n_2, \dots, n_L\rangle$, where $n_i$ is the occupation number. For fermions (corresponding to wavefunctions that are anti-symmetric under particle exchange), the max occupancy of a given site, $i$ is $n_i = 1$. To preserve the symmetric properties of Fock states, second quantization introduces fermionic creation and annihilation operators. The creation operator, $a^{\dagger}_{i}$ creates a particle at site $i$ if unoccupied, and the annihilation operator, $a_i$, removes a particle at site $i$ if occupied. More formally, it's action on a Fock state may be written as, 
\begin{align*}
    a_j^{\dagger} |n_0, \dots, n_j, \dots, n_{m-1}\rangle = {(-1)}^{\sum^{j-1}_{s=0}n_s} (1-n_j) |n_0, \dots, n_j - 1, \dots, n_{m-1}\rangle, \\
    a_j |n_0, \dots, n_j, \dots, n_{m-1}\rangle = {(-1)}^{\sum^{j-1}_{s=0}n_s} n_j |n_0, \dots, n_j - 1, \dots, n_{m-1}\rangle,
\end{align*}

%possible appendix here 




\subsection{Entanglement in Fermionic Systems}