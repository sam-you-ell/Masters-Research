

\subsection{Qubit Systems}
\subsubsection{Quantum Bits} % add some detail about physical relevance and distinguishibility.

In the theory of quantum computation, a quantum bit (qubit) is a spin-$\frac{1}{2}$ particle or a two-level system, in which a single bit of information can be encoded. 
Unlike it's classical counterpart, where a bit occupies a binary state of 0 or 1, a qubit exists in a linear superposition of quantum states, expressed as 
$|\psi\rangle = \alpha |0\rangle + \beta |1\rangle$, where $\alpha$ and $\beta$ are complex probability amplitudes that satisfy $|\alpha|^2 + |\beta|^2 = 1$.
The states $|0\rangle$ and $|1\rangle$ form an orthonormal basis in the simplest Hilbert space, $\mathbb{C}^2$, and are known as computational basis states. 
To extend this description to a system with 2 or more qubits, the use of the tensor product is required. For example, consider two subsystems
$A$ and $B$, with their respective Hilbert spaces, $\mathcal{H}_{A}$ and $\mathcal{H}_{B}$ such that they each describe a single qubit. 
The total Hilbert space, $\mathcal{H}_{AB}$ for the two-qubit, is constructed from $\mathcal{H}_{A}$ and $\mathcal{H}_{B}$, as
\begin{equation}
    \mathcal{H}_{AB} = \mathcal{H}_{A} \otimes \mathcal{H}_{A}.
\end{equation}
To generalise, the Hilbert space of an $n$ qubit system is written as, 
\begin{equation}\label{Hilbert Space}
    \mathcal{H} = \mathcal{H}^{\otimes n} \equiv \mathcal{H}_{1} \otimes \mathcal{H}_{2} \otimes \dots \otimes \mathcal{H}_{n}. 
\end{equation}
The many-qubit states that span $\mathcal{H}$ are constructed identically, and are often expressed as a binary strings for a given configuration, 
\begin{equation}\label{nqubit}
    |x_1\rangle \otimes |x_2\rangle \otimes \dots \otimes |x_n\rangle \equiv |x_1 x_2 \dots x_n\rangle.
\end{equation}





\subsubsection{Quantum Circuits}

The overarching aim of this project focuses on the *interesting phenomena*
that can be found in the evolution of quantum many-body systems, such as ($\refeq{Hilbert Space}$). The dynamics of such evolutions are described by the time-evolution operator, which that maps an initial configuration, $|\psi_0\rangle $ to a time-evolved configuration $|\psi(t)\rangle$ as follows, 
\begin{equation}
    |\psi (t)\rangle = U |\psi_0\rangle. 
\end{equation}
Where $U$ is an arbitrary matrix from the unitary group, $U(2^N)$, acting on the total Hilbert space and satisfying $UU^{\dagger} = U^{\dagger}U = I$. 
Conviniently, unitary evolution may be deconstructed into a sequence of linear transformations acting on finite subregions of the Hilbert space, represented as a quantum circuit.
% This results in an intuitive description of many-body dynamics, where evolutions are represented as a circuit diagrams,
The quantum circuit is constructed to act on a set of qubits, called a register, with each time-step corresponding to a specific action by a quantum logic gate.  
This is analogous to classical computation, where circuits are comprised of logic gates acting on bit-strings of information. 
In contrast, quantum logic gates are linear operators that have a distinct matrix representation \footnote{Any linear map between two finite dimensional vector spaces, in this case finite dimensional Hilbert spaces, may be represented as a matrix. }.  

Each notable quantum logic gate has a specified gate symbol, as can be seen in Fig. \ref{Paulis}, allowing the creation of complicated diagrammatic quantum circuitry that can be directly
mapped to simple matrix manipulations. *maybe talk about benefits on computation*

\begin{figure}[h]
    \centering
    \begin{subfloat}[pauliX]{
        \centering
        \begin{quantikz}
            &  \gate{X} 
                &  \qw
        \end{quantikz}
    }
    \end{subfloat}
    \hspace{10pt} 
    \begin{subfloat}[pauliY]{
        \centering
        \begin{quantikz}
            & \gate{Y}
                & \qw
        \end{quantikz}
    }
    \end{subfloat}
    \hspace{10pt} 
    \begin{subfloat}[pauliZ]{
        \centering
        \begin{quantikz}
            & \gate{Z}
                & \qw
        \end{quantikz}
    }
    \end{subfloat}
\end{figure}

The gates shown in Fig. \ref{Paulis} are  the Pauli operators, equivalent
to the set of Pauli matrices, $P \equiv \{X, Y, Z\}$ for which $X, Y \text{ and } Z$ are defined in their matrix representation as
\begin{align}
    \label{PauliMatrices}
    X = \begin{bmatrix}
            0 & 1 \\
            1 & 0
        \end{bmatrix},
     &  &
    Y = \begin{bmatrix}
            0  & -i \\
            i & 0
        \end{bmatrix},
     &  &
    Z = \begin{bmatrix}
            1 & 0  \\
            0 & -1
        \end{bmatrix},
\end{align}

These gates are all one-qubit gates, as they only act upon a single qubit.
Together with the Identity operator, $I$, the Pauli matrices form an algebra,
satisfying the following relations:
\begin{align}
    XY = iZ,  &  & YZ = iX,  &  & ZX = iY,  \\
    YX = -iZ, &  & ZY = -iX, &  & XZ = -iY,
\end{align}
\begin{align}
    X^2 = Y^2 = Z^2 = I.
\end{align}


The set of Pauli matrices and the identity form
the Pauli group, ${\cal P}_n$, defined as the $4^n$ $n$-qubit tensor products of the Pauli matrices (\ref{PauliMatrices}) and the
Identity matrix, $I$, with multiplicative factors, $\pm 1$ and $\pm i$ to ensure a legitimate group is formed under multiplication.
For clarity, consider the Pauli group on 1-qubit, ${\cal P}_1$:
\begin{equation}\label{PauliGroup}
    {\cal P}_1 \equiv \{ \pm I, \pm iI, \pm X, \pm iX \pm Y, \pm iY, \pm Z, \pm iZ\}.
\end{equation}



From this, another group of interest can be defined, namely the Clifford group, ${\cal C}_n$, defined as a
subset of unitary operators that normalise the Pauli group *INSERT CLIFFORD GROUP DEFINITION*.
Notable elements of this group are the Hadamard, Controlled-Not and Phase operators.

The Hadamard operator, $H$ maps computational basis states to a superposition of computational basis states, written explicitly 
in it's action as, 
\begin{align*}
    H|0\rangle = \frac{|0\rangle + |1\rangle}{\sqrt{2}}, && H|1\rangle = \frac{|0\rangle - |1\rangle}{\sqrt{2}},
\end{align*}
or in matrix form, 
\begin{equation}
    H = \frac{1}{\sqrt{2}} \begin{bmatrix}
        1 & 1\\
        1 & -1
    \end{bmatrix}.
\end{equation}

Controlled-NOT, $CNOT_{AB}$, is a controlled two-qubit gate. The first qubit, $A$ acts as a `control' for an operation to be 
performed on the target qubit, $B$. It's matrix representation is, 
\begin{align*}
    CNOT_{12} = \begin{bmatrix}
        1 & 0 & 0 & 0 \\
        0 & 1 & 0 & 0 \\
        0 & 0 & 0 & 1 \\
        0 & 0 & 1 & 0
        \end{bmatrix}
\end{align*}

The Phase operator, denoted $R$ is defined as,
\begin{align*}
    R =
    \begin{bmatrix}
        1 & 0                  \\
        0 & e^{i\frac{\pi}{2}}
    \end{bmatrix}.
\end{align*}






% \subsection{Entanglement in Qubit Systems}

% The CNOT operator is often used to an generate entangled state. One such state is the maximally entangled 2-qubit state,
% called a Bell state, $|{\bm\Phi}^{+}\rangle_ = (|00\rangle + |11\rangle)/\sqrt{2}$. This is prepared
% from a $|00\rangle$ state, by applying a Hadamard to the first qubit, and subsequently a Controlled-Not gate:
% \begin{itemize}
%     \item[I.] $H \otimes I |00\rangle = \left (\frac{|0\rangle + |1\rangle }{\sqrt{2}}\right )|0\rangle$ 
%     \item[II.]  $CNOT \left (\frac{|0\rangle + |1\rangle }{\sqrt{2}}\right )|0\rangle = \frac{|00\rangle + |11\rangle}{\sqrt{2}}$
% \end{itemize}
% % \begin{align}
% %     H \otimes I |00\rangle = \left (\frac{|0\rangle + |1\rangle }{\sqrt{2}}\right )|0\rangle \\
% %     CNOT \left (\frac{|0\rangle + |1\rangle }{\sqrt{2}}\right )|0\rangle = \frac{|00\rangle + |11\rangle}{\sqrt{2}}
% % \end{align}
% The corresponding circuit representation of this preparation is given in Fig. \ref{Bellstate}.

% \begin{figure}
    \centering
    \begin{quantikz}
        \lstick{$\ket{0}$} & \gate{H} & \ctrl{1} & \qw\rstick[wires=2]{$\ket{\Phi^+}$} \\
        \lstick{$\ket{0}$}& \qw & \targ{} & \qw
    \end{quantikz}
    \caption{Preparation of a Bell state from $\ket{0}$ using a Hadamard and CNOT.}
    \label{Bellstate}
\end{figure}
% \cite{nielsen_chuang_2010}.


% To give an example, consider the preparation of a GHZ state, $\frac{|000\rangle + |111\rangle}{\sqrt{2}}$ 
% from an initial all-zero state, $|000\rangle$. This transformation may be written as, 
% \begin{align*}
%    |\psi_{GHZ}\rangle = U |000\rangle,\\
%    |\psi_{GHZ}\rangle = (CNOT_{13})(CNOT_{12})(H\otimes I \otimes I)|000\rangle. 
% \end{align*}
% Where $I$ is the identity operator. As a circuit diagram, the transformation takes the following form: 
%    \begin{center}
%     \begin{quantikz}
%         \lstick{$\ket{0}$} & \gate{H} & \ctrl{1} & \ctrl{2} & \qw\rstick[wires=3]{$\ket{\psi_{GHZ}}$} \\
%         \lstick{$\ket{0}$}& \qw & \targ{} & \qw & \qw\\
%         \lstick{$\ket{0}$}& \qw & \qw & \targ{} & \qw
%     \end{quantikz}
%     % \caption{Preparation of a Bell state from $\ket{0}$ using a Hadamard and CNOT.}
% \end{center}

%%%%%%%%%%%%%%%






%can perform operations on qubits in a similar fashion to logical operations on bits, since maps are linear we can express them as matrices. introduce some operators. 
%then explain how we may form a quantum circuit from these. 
%then explain how we encode information

