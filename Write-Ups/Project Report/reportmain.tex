\documentclass[nofootinbib]{revtex4-2}

\usepackage{graphicx}
\usepackage[dvipsnames]{xcolor}
\usepackage{tikz}
\usetikzlibrary{quantikz}
\usepackage{amsmath, amsthm, amssymb, mathtools}
\usepackage{fancyhdr}
\usepackage{geometry}
\usepackage{bm}
\usepackage{subfig}
\usepackage{multirow}
\usepackage{array}
\usepackage{hyperref}

\newtheorem{theorem}{Theorem}

\pagestyle{plain}




% \fancyheadoffset{-0.001\textwidth}
% \pagestyle{fancy}
\setlength{\textwidth}{7in}
\setlength{\evensidemargin}{-0.2in}
\setlength{\oddsidemargin}{-0.2in}
\setlength{\headheight}{14pt}
%\setlength{\headwidth}{6in}
\setlength{\topmargin}{-0.5in}
\setlength{\textheight}{8.4in}
% \setlength{\baselineskip}{10pt}


\begin{document}

\title{Quantum Scrambling}

\author{Samuel A. Hopkins$^{1*}$}
\affiliation{$^1$H. H. Wills Physics Laboratory, University of Bristol, Bristol, BS8 1TL, UK}
\date{\today}
\thanks{Email: cn19407@bristol.ac.uk\\
Supervisor: Stephen R. Clark\\
Word Count: TBA
}
\begin{abstract}
    Using atypical quantum circuit models, specifically super-Clifford and non-interacting fermion circuits, we explore the dynamics of operator space entanglement entropy and the process known as scrambling. In super-Clifford circuits, we reproduce the work from Blake and Linden, and achieve a maximal entanglement entropy using the stabilizer formalism. We find that no local operators can generate non-trivial entanglement dynamics within super-Clifford circuits. By using non-interacting fermion circuits we show that fermionic systems subject to random unitary circuits exhibit weak entangling dynamics in operator space, with it's entanglement entropy saturating at a fraction of the Page value.
\end{abstract}
\maketitle

\clearpage
\tableofcontents



%%INTRODUCTION Outline:
% Introduce quantum scrambling - short short description
%where and why is it being studied 
% 
%
\section{Report Plan}
    \subsection{Introduction}
        \begin{itemize}
            \color{ForestGreen}
            \item Introduce quantum scrambling with a short description
            \item Where and why is this being studied
            \item What are the uses and conclusions we can draw - state examples 
            \item What are our aims
            \item Outline the structure of the report
        \end{itemize}


    \subsection{Background Theory}
        \subsubsection{Qubit Systems}
            \begin{itemize}
                \color{ForestGreen}
                \item What are qubits?
                \item Why do we use qubits?
                \item Quantum operators + Quantum Circuits
                \item Entanglement In qubit systems
            \end{itemize}

        \subsubsection{Fermionic Systems}
        \begin{itemize}\color{ForestGreen}
            \item What are fermions? - Introduce Building blocks
            \item Fock space and states; creation and annihilation operators; anticommutation relations
            \item Correlation Functions + Wicks theorem
            \item Majorana Fermions 
            \item Entanglement in Fermionic Systems
        \end{itemize}
        \subsubsection{Encoding Information}
        \begin{itemize}\color{ForestGreen}
            \item How do we encode information? Introduce Pauli strings and why they're a convinient way to encode information. 
            \item Ask stephen about this. 
        \end{itemize}
        
    \subsection{Quantum Scrambling}
        \subsubsection{Operator Spreading}
        \begin{itemize}\color{ForestGreen}
            \item Detailed explanations
            \item Introduce the picture of Spreading
            \item How we measure spreading in full generality
            \item Here we can focus on literature on spreading (since it is abundant)
        \end{itemize}
        \subsubsection{Entanglement Entropy}
        \begin{itemize}\color{ForestGreen}
            \item Explain how a system generates entanglement and how this looks in terms of Pauli strings
            \item Start with shannon entropy as a measure for loss of information 
            \item introduce von neumann entropy and the conditions for an entropy we want 
            \item also introduce renyi but state not used.
            \item 
        \end{itemize}



        \subsection{Appendix}
        \begin{itemize}\color{ForestGreen}
            \item Density Matrices
            \item Wicks theorem
            \item Schur Decomposition
            \item Group Theory (Generators)
            \item Clifford Group
        \end{itemize}






\cleardoublepage

\section{Introduction}


%Theory Outline
%Qubit systems
%- What are qubits and information about the system they occupy
%- 
\section{Qubit Systems}
\subsection{Quantum Bits}
An intuitive example of a quantum many-body system, is that of a quantum computer. In a similar fashion to how a `bit' is the basic unit of 
information within classical computation and logic, the `qubit' (quantum-bit) forms the basic unit of information within quantum computation. 
Drawing the analogy from classical computation, where a bit occupies a binary state of 0 or 1, a quantum bit is a quantum state, denoted $|0\rangle$ or $|1\rangle$. 
These states form an orthonormal basis in $\mathbb{C}^2$ and are known as computational basis states. However, the analogy with classical computing ends here, as qubits can be in a linear superposition of states,
$|\psi\rangle = \alpha |0\rangle + \beta |1\rangle$, where $\alpha, \beta$ are complex probability amplitudes. 

To describe a system of many qubits, the use of the tensor product is required. The Hilbert space of an $n$ qubit system
is constructed via the tensor product of subsystem Hilbert spaces for each qubit, 
\begin{equation}
    \mathcal{H} = \mathcal{H}^{\otimes n} \equiv \mathcal{H}_{1} \otimes \mathcal{H}_{2} \otimes \dots \otimes \mathcal{H}_{n}.
\end{equation}
The state of an $n$ qubit system is constructed identically, and is often expressed as a binary string, 

\begin{equation}
    |x_1\rangle \otimes |x_2\rangle \otimes \dots \otimes |x_n\rangle \equiv |x_1 x_2 \dots x_n\rangle.
\end{equation}

\subsection{Circuits}

To evolve a many-body state, such as an $n$ qubit state, the time-evolution operator, $U$ is used in the following way, 


The evolution and dynamics of many-body systems can be represented via quantum circuits, constructed from 
a set of quantum logic gates acting upon the qubits of a system. Analogous to a classical computer which 
is comprised of logic gates that act upon bit-strings of information.
In contrast, quantum logic gates are linear operators acting on qubits. This allows for the decomposition 
of a unitary evolution into a sequence of linear transformations, represented as matrices\footnote{Any linear map between two finite dimensional vector spaces, in this case finite dimensional Hilbert spaces, may be represented as a matrix. }. 
Such evolutions are often represented as a circuit diagrams, with each time step in the unitary evolution corresponding to a gate action upon a set of qubits.
Analogous to circuit diagrams in
classical computation, each quantum logic gate has a specified gate symbol, allowing the creation of complicated quantum circuitry that can be directly
mapped to a sequence of linear transformations acting on a finite set of qubits.
Some example gate symbols can be seen in Fig. \ref{Paulis}.

\begin{figure}[h]
    \centering
    \begin{subfloat}[pauliX]{
        \centering
        \begin{quantikz}
            &  \gate{X} 
                &  \qw
        \end{quantikz}
    }
    \end{subfloat}
    \hspace{10pt} 
    \begin{subfloat}[pauliY]{
        \centering
        \begin{quantikz}
            & \gate{Y}
                & \qw
        \end{quantikz}
    }
    \end{subfloat}
    \hspace{10pt} 
    \begin{subfloat}[pauliZ]{
        \centering
        \begin{quantikz}
            & \gate{Z}
                & \qw
        \end{quantikz}
    }
    \end{subfloat}
\end{figure}










\subsection{Encoding Information}

%can perform operations on qubits in a similar fashion to logical operations on bits, since maps are linear we can express them as matrices. introduce some operators. 
%then explain how we may form a quantum circuit from these. 
%then explain how we encode information

\section{Fermionic Systems}

\section{Efficient Quantum Circuits}

\section{Quantum Scrambling}
\section{Results}


\bibliographystyle{ieeetr}
\bibliography{report_bibliography.bib}



\end{document}