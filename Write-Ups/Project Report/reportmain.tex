\documentclass[ aps]{revtex4-2}

\usepackage{graphicx}
\usepackage[dvipsnames]{xcolor}
\usepackage{tikz}
\usetikzlibrary{quantikz}
\usepackage{amsmath, amsthm, amssymb, mathtools, bbold}
\usepackage{fancyhdr}
\usepackage{geometry}
\usepackage{bm}
\usepackage{subfig}
\usepackage{multirow}
\usepackage{array}
\usepackage{hyperref}

\newtheorem{theorem}{Theorem}

\pagestyle{plain}




% \fancyheadoffset{-0.001\textwidth}
% \pagestyle{fancy}
\setlength{\textwidth}{7in}
\setlength{\evensidemargin}{-0.2in}
\setlength{\oddsidemargin}{-0.2in}
\setlength{\headheight}{14pt}
%\setlength{\headwidth}{6in}
\setlength{\topmargin}{-0.5in}
\setlength{\textheight}{8.4in}
% \setlength{\baselineskip}{10pt}


\begin{document}

\title{Quantum Scrambling}

\author{Samuel A. Hopkins$^{1*}$}
\affiliation{$^1$H. H. Wills Physics Laboratory, University of Bristol, Bristol, BS8 1TL, UK}
\date{\today}
\thanks{Email: cn19407@bristol.ac.uk\\
Supervisor: Stephen R. Clark\\
Word Count: TBA
}
\begin{abstract}
    Using atypical quantum circuit models, specifically super-Clifford and non-interacting fermion circuits, we explore the dynamics of operator space entanglement entropy and the process known as scrambling. In super-Clifford circuits, we reproduce the work from Blake and Linden, and achieve a maximal entanglement entropy using the stabilizer formalism. We find that no local operators can generate non-trivial entanglement dynamics within super-Clifford circuits. By using non-interacting fermion circuits we show that fermionic systems subject to random unitary circuits exhibit weak entangling dynamics in operator space, with it's entanglement entropy saturating at a fraction of the Page value.
\end{abstract}
\maketitle

\clearpage
\tableofcontents



%%INTRODUCTION Outline:
% Introduce quantum scrambling - short short description
%where and why is it being studied 
% 
%
% \section{Report Plan}
%     \subsection{Introduction}
%         \begin{itemize}
%             \color{ForestGreen}
%             \item Introduce quantum scrambling with a short description
%             \item Where and why is this being studied
%             \item What are the uses and conclusions we can draw - state examples 
%             \item What are our aims
%             \item Outline the structure of the report
%         \end{itemize}


%     \subsection{Background Theory}
%         \subsubsection{Qubit Systems}
%             \begin{itemize}
%                 \color{ForestGreen}
%                 \item What are qubits?
%                 \item Why do we use qubits?
%                 \item Quantum operators + Quantum Circuits
%                 \item Entanglement In qubit systems
%             \end{itemize}

%         \subsubsection{Fermionic Systems}
%         \begin{itemize}\color{ForestGreen}
%             \item What are fermions? - Introduce Building blocks
%             \item Fock space and states; creation and annihilation operators; anticommutation relations
%             \item Correlation Functions + Wicks theorem
%             \item Majorana Fermions 
%             \item Entanglement in Fermionic Systems
%         \end{itemize}
%         \subsubsection{Encoding Information}
%         \begin{itemize}\color{ForestGreen}
%             \item How do we encode information? Introduce Pauli strings and why they're a convinient way to encode information. 
%             \item Ask stephen about this. 
%         \end{itemize}
        
%     \subsection{Quantum Scrambling}
%         \subsubsection{Operator Spreading}
%         \begin{itemize}\color{ForestGreen}
%             \item Detailed explanations
%             \item Introduce the picture of Spreading
%             \item How we measure spreading in full generality
%             \item Here we can focus on literature on spreading (since it is abundant)
%         \end{itemize}
%         \subsubsection{Entanglement Entropy}
%         \begin{itemize}\color{ForestGreen}
%             \item Explain how a system generates entanglement and how this looks in terms of Pauli strings
%             \item Start with shannon entropy as a measure for loss of information 
%             \item introduce von neumann entropy and the conditions for an entropy we want 
%             \item also introduce renyi but state not used.
%             \item 
%         \end{itemize}



%         \subsection{Appendix}
%         \begin{itemize}\color{ForestGreen}
%             \item Density Matrices
%             \item Wicks theorem
%             \item Schur Decomposition
%             \item Group Theory (Generators)
%             \item Clifford Group
%         \end{itemize}





\cleardoublepage



\section{Introduction}
Many-body systems and their dynamics play a central role in our understanding of modern physics with the
dynamics of quantum many-body systems offering a great insight to a wide variety of fields in contemporary physics. Studying the evolution of quantum many-body systems presents challenging problems and has provided insightful results in condensed matter physics and quantum information \cite{Polkovnikov_2011}.
Studying the dynamics of such systems is fundamentally
a computational challenge due to the exponential growth of the Hilbert space with the number of
qubits. However, quantum dynamics can be efficiently simulated in atypical quantum circuit models,
providing a rich theoretical playground to test our understanding and uncover interesting phenomena.
These atypical circuit models, known as classically simulable circuit models, 
will be analysed in hopes of understanding the spreading and scrambling of encoded local information,
a process known as quantum scrambling. More precisely, quantum scrambling describes the
process in which local information encoded by a simple product operator, such as a string of Pauli operators, becomes 'scrambled' by a  amongst the large number of degrees of freedom in the system, such that, the operator becomes a highly complicated sum of product operators.

In recent years, the study of this scrambling process has seen immense research yielded new perspectives. These include new insights into the study of information in black holes and the 
AdS/CFT correspondence \cite{Calabrese_2009,Jensen_2016,Sekino_2008,ShenkerBlackHolesButterfly2014},
quantum chaos, \cite{Maldacena_2016, https://doi.org/10.48550/arxiv.1412.6087} and operator hydrodynamics in many-body systems \cite{Khemani_2018, PhysRevX.8.021013, PhysRevX.8.031058,Grozdanov_2018}.

In order to investigate this phenomenon,classically simulable models provide excellent tools to gain a deeper understanding of operator evolution in hopes of applying any findings
to generic cases of unitary evolution in many-body systems.
As previously mentioned, some families of circuits are able to be efficiently
simulated with polynomial effort on a classical computer. Two well-studied circuit models take the principal interest of this review, namely Clifford circuits and non-interacting fermion circuits. Clifford circuits and the Clifford
group have played a key role in the study fault-tolerant quantum computing and, more recently, many-body physics via
random Clifford unitaries \cite{PhysRevB.98.205136, https://doi.org/10.48550/arxiv.2110.02988}.


This is due to their efficient simulability on classical computers, despite the high amounts of entanglement that can be generated amongst states. Hence, circuits such as Clifford circuit are an ideal candidate in the study of many-body systems and their dynamics \cite{https://doi.org/10.48550/arxiv.2210.10129}.

The second class of circuit models to be explored, are Matchgate or non-interacting fermion circuits. First introduced by Valiant \cite{Valiant2001QuantumCT} in the matchgate formalism and later identified with a physical model of non-interacting fermions in one dimension. This system has been shown to be classically simulable by DiVincenzo and Terhal \cite{Terhal2001} as an extension of Valiant's findings. Thus, free fermion circuits form a good candidate to study scrambling phenomena, and the limitations of these circuit models.

With the use of recently developed techniques, the key objective of this project is to examine 
operator scrambling in models where the dynamics are analytically and numerically tractable. This is not
possible in the generic unitary evolution of many-body systems due to an exponential growth of
the Hilbert space with system size. However, by simulating the dynamics of scrambling in simulable circuit models, a picture of unitary evolution can be constructed, in hopes of gaining an understanding of dynamics of operator evolution and entanglement in atypical quantum systems. 








%Theory Outline
%Qubit systems
%- What are qubits and information about the system they occupy
%- 


\subsection{Qubit Systems}
\subsubsection{Quantum Bits} % add some detail about physical relevance and distinguishibility.

In the theory of quantum computation, a quantum bit (qubit) is a spin-$\frac{1}{2}$ particle or a two-level system, in which a single bit of information can be encoded. 
Unlike it's classical counterpart, where a bit occupies a binary state of 0 or 1, a qubit exists in a linear superposition of quantum states, expressed as 
$|\psi\rangle = \alpha |0\rangle + \beta |1\rangle$, where $\alpha$ and $\beta$ are complex probability amplitudes that satisfy $|\alpha|^2 + |\beta|^2 = 1$.
The states $|0\rangle$ and $|1\rangle$ form an orthonormal basis in the simplest Hilbert space, $\mathbb{C}^2$, and are known as computational basis states. 
To extend this description to a system with 2 or more qubits, the use of the tensor product is required. For example, consider two subsystems
$A$ and $B$, with their respective Hilbert spaces, $\mathcal{H}_{A}$ and $\mathcal{H}_{B}$ such that they each describe a single qubit. 
The total Hilbert space, $\mathcal{H}_{AB}$ for the two-qubit, is constructed from $\mathcal{H}_{A}$ and $\mathcal{H}_{B}$, as
\begin{equation}
    \mathcal{H}_{AB} = \mathcal{H}_{A} \otimes \mathcal{H}_{A}.
\end{equation}
To generalise, the Hilbert space of an $n$ qubit system is written as, 
\begin{equation}\label{Hilbert Space}
    \mathcal{H} = \mathcal{H}^{\otimes n} \equiv \mathcal{H}_{1} \otimes \mathcal{H}_{2} \otimes \dots \otimes \mathcal{H}_{n}. 
\end{equation}
The many-qubit states that span $\mathcal{H}$ are constructed identically, and are often expressed as a binary strings for a given configuration, 
\begin{equation}\label{nqubit}
    |x_1\rangle \otimes |x_2\rangle \otimes \dots \otimes |x_n\rangle \equiv |x_1 x_2 \dots x_n\rangle.
\end{equation}





\subsubsection{Quantum Circuits}

The overarching aim of this project focuses on the *interesting phenomena*
that can be found in the evolution of quantum many-body systems, such as ($\refeq{Hilbert Space}$). The dynamics of such evolutions are described by the time-evolution operator, which that maps an initial configuration, $|\psi_0\rangle $ to a time-evolved configuration $|\psi(t)\rangle$ as follows, 
\begin{equation}
    |\psi (t)\rangle = U |\psi_0\rangle. 
\end{equation}
Where $U$ is an arbitrary matrix from the unitary group, $U(2^N)$, acting on the total Hilbert space and satisfying $UU^{\dagger} = U^{\dagger}U = I$. 
Conviniently, unitary evolution may be deconstructed into a sequence of linear transformations acting on finite subregions of the Hilbert space, represented as a quantum circuit.
% This results in an intuitive description of many-body dynamics, where evolutions are represented as a circuit diagrams,
The quantum circuit is constructed to act on a set of qubits, called a register, with each time-step corresponding to a specific action by a quantum logic gate.  
This is analogous to classical computation, where circuits are comprised of logic gates acting on bit-strings of information. 
In contrast, quantum logic gates are linear operators that have a distinct matrix representation \footnote{Any linear map between two finite dimensional vector spaces, in this case finite dimensional Hilbert spaces, may be represented as a matrix. }.  

Each notable quantum logic gate has a specified gate symbol, as can be seen in Fig. \ref{Paulis}, allowing the creation of complicated diagrammatic quantum circuitry that can be directly
mapped to simple matrix manipulations. *maybe talk about benefits on computation*

\begin{figure}[h]
    \centering
    \begin{subfloat}[pauliX]{
        \centering
        \begin{quantikz}
            &  \gate{X} 
                &  \qw
        \end{quantikz}
    }
    \end{subfloat}
    \hspace{10pt} 
    \begin{subfloat}[pauliY]{
        \centering
        \begin{quantikz}
            & \gate{Y}
                & \qw
        \end{quantikz}
    }
    \end{subfloat}
    \hspace{10pt} 
    \begin{subfloat}[pauliZ]{
        \centering
        \begin{quantikz}
            & \gate{Z}
                & \qw
        \end{quantikz}
    }
    \end{subfloat}
\end{figure}

The gates shown in Fig. \ref{Paulis} are  the Pauli operators, equivalent
to the set of Pauli matrices, $P \equiv \{X, Y, Z\}$ for which $X, Y \text{ and } Z$ are defined in their matrix representation as
\begin{align}
    \label{PauliMatrices}
    X = \begin{bmatrix}
            0 & 1 \\
            1 & 0
        \end{bmatrix},
     &  &
    Y = \begin{bmatrix}
            0  & -i \\
            i & 0
        \end{bmatrix},
     &  &
    Z = \begin{bmatrix}
            1 & 0  \\
            0 & -1
        \end{bmatrix},
\end{align}

These gates are all one-qubit gates, as they only act upon a single qubit.
Together with the Identity operator, $I$, the Pauli matrices form an algebra,
satisfying the following relations:
\begin{align}
    XY = iZ,  &  & YZ = iX,  &  & ZX = iY,  \\
    YX = -iZ, &  & ZY = -iX, &  & XZ = -iY,
\end{align}
\begin{align}
    X^2 = Y^2 = Z^2 = I.
\end{align}


The set of Pauli matrices and the identity form
the Pauli group, ${\cal P}_n$, defined as the $4^n$ $n$-qubit tensor products of the Pauli matrices (\ref{PauliMatrices}) and the
Identity matrix, $I$, with multiplicative factors, $\pm 1$ and $\pm i$ to ensure a legitimate group is formed under multiplication.
For clarity, consider the Pauli group on 1-qubit, ${\cal P}_1$:
\begin{equation}\label{PauliGroup}
    {\cal P}_1 \equiv \{ \pm I, \pm iI, \pm X, \pm iX \pm Y, \pm iY, \pm Z, \pm iZ\}.
\end{equation}



From this, another group of interest can be defined, namely the Clifford group, ${\cal C}_n$, defined as a
subset of unitary operators that normalise the Pauli group *INSERT CLIFFORD GROUP DEFINITION*.
Notable elements of this group are the Hadamard, Controlled-Not and Phase operators.

The Hadamard operator, $H$ maps computational basis states to a superposition of computational basis states, written explicitly 
in it's action as, 
\begin{align*}
    H|0\rangle = \frac{|0\rangle + |1\rangle}{\sqrt{2}}, && H|1\rangle = \frac{|0\rangle - |1\rangle}{\sqrt{2}},
\end{align*}
or in matrix form, 
\begin{equation}
    H = \frac{1}{\sqrt{2}} \begin{bmatrix}
        1 & 1\\
        1 & -1
    \end{bmatrix}.
\end{equation}

Controlled-NOT, $CNOT_{AB}$, is a controlled two-qubit gate. The first qubit, $A$ acts as a `control' for an operation to be 
performed on the target qubit, $B$. It's matrix representation is, 
\begin{align*}
    CNOT_{12} = \begin{bmatrix}
        1 & 0 & 0 & 0 \\
        0 & 1 & 0 & 0 \\
        0 & 0 & 0 & 1 \\
        0 & 0 & 1 & 0
        \end{bmatrix}
\end{align*}

The Phase operator, denoted $R$ is defined as,
\begin{align*}
    R =
    \begin{bmatrix}
        1 & 0                  \\
        0 & e^{i\frac{\pi}{2}}
    \end{bmatrix}.
\end{align*}






% \subsection{Entanglement in Qubit Systems}

% The CNOT operator is often used to an generate entangled state. One such state is the maximally entangled 2-qubit state,
% called a Bell state, $|{\bm\Phi}^{+}\rangle_ = (|00\rangle + |11\rangle)/\sqrt{2}$. This is prepared
% from a $|00\rangle$ state, by applying a Hadamard to the first qubit, and subsequently a Controlled-Not gate:
% \begin{itemize}
%     \item[I.] $H \otimes I |00\rangle = \left (\frac{|0\rangle + |1\rangle }{\sqrt{2}}\right )|0\rangle$ 
%     \item[II.]  $CNOT \left (\frac{|0\rangle + |1\rangle }{\sqrt{2}}\right )|0\rangle = \frac{|00\rangle + |11\rangle}{\sqrt{2}}$
% \end{itemize}
% % \begin{align}
% %     H \otimes I |00\rangle = \left (\frac{|0\rangle + |1\rangle }{\sqrt{2}}\right )|0\rangle \\
% %     CNOT \left (\frac{|0\rangle + |1\rangle }{\sqrt{2}}\right )|0\rangle = \frac{|00\rangle + |11\rangle}{\sqrt{2}}
% % \end{align}
% The corresponding circuit representation of this preparation is given in Fig. \ref{Bellstate}.

% \begin{figure}
    \centering
    \begin{quantikz}
        \lstick{$\ket{0}$} & \gate{H} & \ctrl{1} & \qw\rstick[wires=2]{$\ket{\Phi^+}$} \\
        \lstick{$\ket{0}$}& \qw & \targ{} & \qw
    \end{quantikz}
    \caption{Preparation of a Bell state from $\ket{0}$ using a Hadamard and CNOT.}
    \label{Bellstate}
\end{figure}
% \cite{nielsen_chuang_2010}.


% To give an example, consider the preparation of a GHZ state, $\frac{|000\rangle + |111\rangle}{\sqrt{2}}$ 
% from an initial all-zero state, $|000\rangle$. This transformation may be written as, 
% \begin{align*}
%    |\psi_{GHZ}\rangle = U |000\rangle,\\
%    |\psi_{GHZ}\rangle = (CNOT_{13})(CNOT_{12})(H\otimes I \otimes I)|000\rangle. 
% \end{align*}
% Where $I$ is the identity operator. As a circuit diagram, the transformation takes the following form: 
%    \begin{center}
%     \begin{quantikz}
%         \lstick{$\ket{0}$} & \gate{H} & \ctrl{1} & \ctrl{2} & \qw\rstick[wires=3]{$\ket{\psi_{GHZ}}$} \\
%         \lstick{$\ket{0}$}& \qw & \targ{} & \qw & \qw\\
%         \lstick{$\ket{0}$}& \qw & \qw & \targ{} & \qw
%     \end{quantikz}
%     % \caption{Preparation of a Bell state from $\ket{0}$ using a Hadamard and CNOT.}
% \end{center}

%%%%%%%%%%%%%%%






%can perform operations on qubits in a similar fashion to logical operations on bits, since maps are linear we can express them as matrices. introduce some operators. 
%then explain how we may form a quantum circuit from these. 
%then explain how we encode information



\subsection{Fermionic Systems}
%Indistinguishibility. 

The familiar qubit system may be mapped onto a system of identical particles (fermions), such that the 
overall many body state describing the system, is invariant under particle exchange. This is 
performed via a Jordan-Wigner transformation, which maps any local spin-model to a local fermionic model.
To gain an understanding of how this can be carried out and why it is relevant, it will be useful to introduce the core concepts and language from \textit{Second Quantization}.

\subsubsection{Second Quantization and Indistinguishable Particles}


The wave function for a system of $N$ identical particles is $\psi(x_1, x_2, \dots x_N)$, where a particle is specified by it's position vector, $\vec{x}$. For bosonic systems, the wavefunction is symmetric under particle exhchange, while fermionic systems present anti-symmetric wavefunctions under particle exchange. 

% For a specified quantum many-body state, wherein the particles are identical, if two particles are exchanged the initial many-body state will remain unchanged.
% (This is different) to classical mechanics, where if particles' position vectors are exchanged, the resulting many-body state is an entirely new configuration. 

The construction of the $n$ particle state via the extension of the single particle wavefunction, as described in (\refeq{nqubit}), 
leads to a redundancy in it's description of a many-body state and an unnecessarily large Hilbert space. 



A more efficient appraoch to describing many-body states is the formalism of second quantisation.Instead of describing states with Slater determinants, a \textit{Fock state} presents an elegant and convinient basis in which to work in. The Fock state of a many-body system is represented in an occupancy number basis, written as $|n_1, n_2, \dots, n_L\rangle$, where $n_i$ is the occupation number. For wavefunctions that are symmetric under particle exchange, i.e bosonic systems, the occupancy can take any real non-negative integer. For wavefunctions that are anti-symmetric under particle exchange (fermionic systems), the occupancy number is either $0$ or $1$ such that the state satisfies the Pauli exclusion principle.

To preserve the symmetric properties of Fock states, second quantization introduces fermionic creation and annihilation operators. The creation operator, $a^{\dagger}_{i}$ creates a particle at site $i$ if unoccupied, and the annihilation operator, $a_i$, removes a particle at site $i$ if occupied. More formally, it's action on a Fock state may be written as, 
\begin{align}
    a_j^{\dagger} |n_0, \dots, n_j, \dots, n_{m-1}\rangle = {(-1)}^{\sum^{j-1}_{s=0}n_s} (1-n_j) |n_0, \dots, n_j - 1, \dots, n_{m-1}\rangle, \\
    a_j |n_0, \dots, n_j, \dots, n_{m-1}\rangle = {(-1)}^{\sum^{j-1}_{s=0}n_s} n_j |n_0, \dots, n_j - 1, \dots, n_{m-1}\rangle.
\end{align}
Using these operators, an arbitrary Fock state, $|\psi_{F}\rangle$ may be constructed from a set of creation operators acting on the vacuum, 
\begin{equation}
    |\psi_{f}\rangle = (a_{1}^{\dagger})^{n_1} (a_{2}^{\dagger})^{n_2} \dots (a_{L}^{\dagger})^{n_{L}}|{\mathbf 0}\rangle.
\end{equation}
The fermionic creation and annihilation operators obey crucial anti-commutation relations, constructed to 
\begin{align}
    \{a_j, a_k\} \equiv \{a_j^{\dagger}, a_k^{\dagger}\} = 0, && \{a_j, a_k^{\dagger}\} = \delta_{jk}I
\end{align}
As an alternative description, the the \textit{Majorana operators} may be defined as 
$c_{2k} = a_k + a_k^{\dagger}$ and $c_{2k+1} = i(a_k^{\dagger} - a_{k})$ for $k =  1, \dots, N$ such that they satisfy $\{c_j, c_k\} = 2\delta_{jk}$. From this, the Jordan-Wigner transformation may be defined, which maps a system of $N$ spin-$\frac{1}{2}$ particles onto a system of $N$ `spinless' fermions as by defining the Majorana operators onto Pauli spin operators: 

\begin{align}
    c
\end{align}
Such a mapping allows the Hilbert space of $N$ qubits to be identified with the Hilbert space of $N$ local fermionic modes (LFM's). 
%introduce majorana operators


%possible appendix here 





% 
The first system to be investigated was the Stabilizer circuits constructed by Blake and Linden \cite{Blake2020}. The system is encoded in an $N \times 2N$ matrix or stabilizer tableau, initialised for the all-$X$ string, $X_1, X_2, \dots, X_N$. A random unitary circuit, constructed from the gates, $\mathbf{Z.H}$ and $\mathbf{C3}$, for a circuit depth of $t=20000$. The entanglement entropy is calculated at each time step via (refeq) and plotted against time to produce FIG. \ref{Bal}. The maximum entanglement entropy is found to saturate at the Page value \cite{Page_1993}, signalling a maximally entangled system. Note that the inclusion of $\mathbf{SWAP}$ does not affect any entanglement dynamics, and is omitted from the circuit. We also find that any intially local string of operators, e.g $I\dots X \dots I$ does not generate any entanglement within the system.

\begin{figure}[t!]
    \centering
    \includegraphics[width=0.48\textwidth]{reportimages/CliffvsFerm.pdf}
    \caption{Comparison of the operator entanglement entropy as a fraction of the maximum entanglement entropy $S_{\infty}$ for a $N = 120$ qubit super-Clifford circuit and a $L = 120$ site, non-interacting fermion circuit. Clifford circuit is constructed from the $\mathbf{Z.H}$ and $\mathbf{C3}$ gates. The non-interacting fermion circuit is constructed from the $G(i, j)$ gate to act on an initially half filled state corresponding to $M^{\text{block}}$. }
    \label{Bal}
\end{figure}
\begin{figure}[h!]
    \centering
    \includegraphics[width=0.48\textwidth]{reportimages/GFock120.pdf}
    \caption{Operator entanglement entropy as a fraction of the maximum entanglement entropy of a fermionicsystem with $L = 120$ modes subject to a non-interacting fermionic circuit of $G(i, j)$. Initial configuration of the system is the alternately-filled state corresponding to $M^{\text{half}}$.}
    \label{halffig}
\end{figure}



\begin{figure}[h!]
    \centering
    \includegraphics[width=0.48\textwidth]{reportimages/GZFockState120.pdf}
    \caption{Comparison of the operator entanglement entropy as a fraction of the maximum entanglement entropy of fermionic systems with $L = 120$ modes subject to non-interacting fermionic circuits of $G(i, j)$ and $G(i, j, k)$. Initial configuration of the system is the alternately-filled state corresponding to $M^{\text{half}}$.}

    \label{comp}
\end{figure}

\begin{figure}[h!]
    \centering
    \includegraphics[width=0.48\textwidth]{reportimages/varysystem.pdf}
    \caption{Comparison of the operator entanglement entropy for fermionic systems of varying sizes, ($L = 20, 40, 60, 80$). $L$ modes are subject to a random non-interacting fermionic circuits of $G(i, j)$ and $G(i, j, k)$. Initial configuration of the system is the alternately-filled state corresponding to $M^{\text{half}}$.}

    \label{vary}
\end{figure}



To compare with super Clifford circuits, systems of $L = 120$ modes were simulated for two configurations under two random unitary circuits, built from $G(i, j)$ and $G(i, j, k)$. 
Each gate was constructed in a simulation of the full state-level description, in order to verify the correct actions and expressions before using techniques to efficiently simulate the fermionic systems. 
 Each circuit was first confirmed to not create entanglement on the vacuum state, corresponding to $M^{\text{vac}}$, and then subsequently simulated on the different filled configurations $M^{\text{half}}$ and $M^{\text{block}}$. The results from the random unitary circuit built from $G(i, j)$ can be seen in Fig. \ref{halffig} and \ref{blockstate}. We find that for both system, the entanglement entropy does not saturate at the maximum Page value. The alternating half-filled state when subject to a random unitary circuit, saturates at $t\sim 2300$, which is considerably lower than the state corresponding to $M^{\text{block}}$, which saturates at $t \sim 16000$. This difference in saturation times, is concisely presented in Fig. \ref{comp}, where saturation level is identical in both circuits, but the circuit built from $G(i, j)$ saturates significantly earlier. Simulations of varying system size were also carried out, and it can be seen that the entanglement saturation is directly dependent on the size of the system. 
\begin{figure}[t!]
    
    \centering
    \includegraphics[width=0.48\textwidth]{reportimages/GFerm120.pdf}
    \caption{Operator entanglement entropy as a fraction of the maximum entanglement entropy of a fermionicsystem with $L = 120$ modes subject to a non-interacting fermionic circuit of $G(i, j)$. Initial configuration of the system is the half-filled state corresponding to $M^{\text{Block}}$. }
    \label{blockstate}
\end{figure}











\bibliographystyle{ieeetr}
\bibliography{report_bibliography.bib}



\end{document}