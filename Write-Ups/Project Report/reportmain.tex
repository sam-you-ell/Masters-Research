\documentclass[ aps]{revtex4-2}

\usepackage{graphicx}
\usepackage[dvipsnames]{xcolor}
\usepackage{tikz}
\usetikzlibrary{quantikz}
\usepackage{amsmath, amsthm, amssymb, mathtools}
\usepackage{fancyhdr}
\usepackage{geometry}
\usepackage{bm}
\usepackage{subfig}
\usepackage{multirow}
\usepackage{array}
\usepackage{hyperref}

\newtheorem{theorem}{Theorem}

\pagestyle{plain}




% \fancyheadoffset{-0.001\textwidth}
% \pagestyle{fancy}
\setlength{\textwidth}{7in}
\setlength{\evensidemargin}{-0.2in}
\setlength{\oddsidemargin}{-0.2in}
\setlength{\headheight}{14pt}
%\setlength{\headwidth}{6in}
\setlength{\topmargin}{-0.5in}
\setlength{\textheight}{8.4in}
% \setlength{\baselineskip}{10pt}


\begin{document}

\title{Quantum Scrambling}

\author{Samuel A. Hopkins$^{1*}$}
\affiliation{$^1$H. H. Wills Physics Laboratory, University of Bristol, Bristol, BS8 1TL, UK}
\date{\today}
\thanks{Email: cn19407@bristol.ac.uk\\
Supervisor: Stephen R. Clark\\
Word Count: TBA
}
\begin{abstract}
    Using atypical quantum circuit models, specifically super-Clifford and non-interacting fermion circuits, we explore the dynamics of operator space entanglement entropy and the process known as scrambling. In super-Clifford circuits, we reproduce the work from Blake and Linden, and achieve a maximal entanglement entropy using the stabilizer formalism. We find that no local operators can generate non-trivial entanglement dynamics within super-Clifford circuits. By using non-interacting fermion circuits we show that fermionic systems subject to random unitary circuits exhibit weak entangling dynamics in operator space, with it's entanglement entropy saturating at a fraction of the Page value.
\end{abstract}
\maketitle

\clearpage
\tableofcontents



%%INTRODUCTION Outline:
% Introduce quantum scrambling - short short description
%where and why is it being studied 
% 
%
% \section{Report Plan}
%     \subsection{Introduction}
%         \begin{itemize}
%             \color{ForestGreen}
%             \item Introduce quantum scrambling with a short description
%             \item Where and why is this being studied
%             \item What are the uses and conclusions we can draw - state examples 
%             \item What are our aims
%             \item Outline the structure of the report
%         \end{itemize}


%     \subsection{Background Theory}
%         \subsubsection{Qubit Systems}
%             \begin{itemize}
%                 \color{ForestGreen}
%                 \item What are qubits?
%                 \item Why do we use qubits?
%                 \item Quantum operators + Quantum Circuits
%                 \item Entanglement In qubit systems
%             \end{itemize}

%         \subsubsection{Fermionic Systems}
%         \begin{itemize}\color{ForestGreen}
%             \item What are fermions? - Introduce Building blocks
%             \item Fock space and states; creation and annihilation operators; anticommutation relations
%             \item Correlation Functions + Wicks theorem
%             \item Majorana Fermions 
%             \item Entanglement in Fermionic Systems
%         \end{itemize}
%         \subsubsection{Encoding Information}
%         \begin{itemize}\color{ForestGreen}
%             \item How do we encode information? Introduce Pauli strings and why they're a convinient way to encode information. 
%             \item Ask stephen about this. 
%         \end{itemize}
        
%     \subsection{Quantum Scrambling}
%         \subsubsection{Operator Spreading} Can bulk word count out here - for sake of "completeness"
%         \begin{itemize}\color{ForestGreen}
%             \item Detailed explanations
%             \item Introduce the picture of Spreading
%             \item How we measure spreading in full generality
%             \item Here we can focus on literature on spreading (since it is abundant)
%         \end{itemize}
%         \subsubsection{Entanglement Entropy}
%         \begin{itemize}\color{ForestGreen}
%             \item Explain how a system generates entanglement and how this looks in terms of Pauli strings
%             \item Start with shannon entropy as a measure for loss of information 
%             \item introduce von neumann entropy and the conditions for an entropy we want 
%             \item also introduce renyi but state not used.
%             \item 
%         \end{itemize}



%         \subsection{Appendix}
%         \begin{itemize}\color{ForestGreen}
%             \item Density Matrices
%             \item Wicks theorem
%             \item Schur Decomposition
%             \item Group Theory (Generators)
%             \item Clifford Group
%         \end{itemize}


%Images
% Quantum circuit example
% Random UNitary circuit
% Lightcone plot




\cleardoublepage

\section{Introduction}


%Theory Outline
%Qubit systems
%- What are qubits and information about the system they occupy
%- 

\section{Qubit Systems}


\newpage
\section{Qubit Systems}
\subsection{Quantum Bits} % add some detail about physical relevance and distinguishibility.

In the theory of quantum computation, a quantum bit (qubit) is a spin-$\frac{1}{2}$ particle or a two-level system, in which a single bit of information can be encoded. 
Unlike it's classical counterpart, where a bit occupies a binary state of 0 or 1, a qubit exists in a linear superposition of quantum states, expressed as 
$|\psi\rangle = \alpha |0\rangle + \beta |1\rangle$, where $\alpha$ and $\beta$ are complex probability amplitudes.
The states $|0\rangle$ and $|1\rangle$ form an orthonormal basis in the simplest Hilbert space, $\mathbb{C}^2$, and are known as computational basis states. 
To extend this description to a system with 2 or more qubits, the use of the tensor product is required. For example, consider two subsystems
$A$ and $B$, with their respective Hilbert spaces, $\mathcal{H}_{A}$ and $\mathcal{H}_{B}$ such that they each describe a single qubit. 
The total Hilbert space, $\mathcal{H}_{AB}$ for the two-qubit, is constructed from $\mathcal{H}_{A}$ and $\mathcal{H}_{B}$, as
\begin{equation}
    \mathcal{H}_{AB} = \mathcal{H}_{A} \otimes \mathcal{H}_{A}.
\end{equation}
To generalise, the Hilbert space of an $n$ qubit system is written as, 
\begin{equation}
    \mathcal{H} = \mathcal{H}^{\otimes n} \equiv \mathcal{H}_{1} \otimes \mathcal{H}_{2} \otimes \dots \otimes \mathcal{H}_{n}. 
\end{equation}
The many-qubit states that span $\mathcal{H}$ are constructed identically, and are often expressed as a binary strings for a given configuration, 
\begin{equation}\label{nqubit}
    |x_1\rangle \otimes |x_2\rangle \otimes \dots \otimes |x_n\rangle \equiv |x_1 x_2 \dots x_n\rangle.
\end{equation}





\subsection{Quantum Circuits}

The overarching aim of this *report* focuses on how many body systems, e.g an $n$ qubit system, as described above, 
evolves in time. The evolution of a quantum system is described by a unitary transformation, that maps an initial configuration, $|\psi_0\rangle $
to a time-evolved configuration $|\psi\rangle$ as follows, 
\begin{equation}
    |\psi \rangle = U |\psi_0\rangle. 
\end{equation}
Where $U$ is some unitary operator, $UU^{\dagger} = U^{\dagger}U = I$. 
In the context of qubit systems, unitary evolution may be deconstructed into a sequence of linear transformations acting on a finite subregion of the Hilbert space. 
This results in an intuitive description of many-body dynamics, where evolutions are represented as a circuit diagrams, 
with each time step in the unitary evolution corresponding to a quantum logic gate action upon a set of qubits.
This is analogous to classical computation, where circuits are comprised of logic gates acting on bit-strings of information. 
In contrast, quantum logic gates are linear operators that have a distinct matrix representation \footnote{Any linear map between two finite dimensional vector spaces, in this case finite dimensional Hilbert spaces, may be represented as a matrix. }.  

Each quantum logic gate has a specified gate symbol, as can be seen in Fig. \ref{Paulis}, allowing the creation of complicated quantum circuitry that can be directly
mapped to simple matrix manipulations.

\begin{figure}[h]
    \centering
    \begin{subfloat}[pauliX]{
        \centering
        \begin{quantikz}
            &  \gate{X} 
                &  \qw
        \end{quantikz}
    }
    \end{subfloat}
    \hspace{10pt} 
    \begin{subfloat}[pauliY]{
        \centering
        \begin{quantikz}
            & \gate{Y}
                & \qw
        \end{quantikz}
    }
    \end{subfloat}
    \hspace{10pt} 
    \begin{subfloat}[pauliZ]{
        \centering
        \begin{quantikz}
            & \gate{Z}
                & \qw
        \end{quantikz}
    }
    \end{subfloat}
\end{figure}

The gates shown in Fig. \ref{Paulis} are  the Pauli operators, equivalent
to the set of Pauli matrices, $P \equiv \{X, Y, Z\}$ for which $X, Y \text{ and } Z$ are defined in their matrix representation as
\begin{align}
    \label{PauliMatrices}
    X = \begin{bmatrix}
            0 & 1 \\
            1 & 0
        \end{bmatrix},
     &  &
    Y = \begin{bmatrix}
            0  & -i \\
            i & 0
        \end{bmatrix},
     &  &
    Z = \begin{bmatrix}
            1 & 0  \\
            0 & -1
        \end{bmatrix},
\end{align}

These gates are all one-qubit gates, as they only act upon a single qubit.
Together with the Identity operator, $I$, the Pauli matrices form an algebra, such that they 
satisfy the following relations:
\begin{align}
    XY = iZ,  &  & YZ = iX,  &  & ZX = iY,  \\
    YX = -iZ, &  & ZY = -iX, &  & XZ = -iY,
\end{align}
\begin{align}
    X^2 = Y^2 = Z^2 = I.
\end{align}


The set of Pauli matrices and the identity form
the Pauli group, ${\cal P}_n$, defined as the $4^n$ $n$-qubit tensor products of the Pauli matrices (\ref{PauliMatrices}) and the
Identity matrix, $I$, with multiplicative factors, $\pm 1$ and $\pm i$ to ensure a legitimate group is formed under multiplication.
For clarity, consider the Pauli group on 1-qubit, ${\cal P}_1$:
\begin{equation}
    {\cal P}_1 \equiv \{ \pm I, \pm iI, \pm X, \pm iX \pm Y, \pm iY, \pm Z, \pm iZ\}.
\end{equation}



From this, another group of interest can be defined, namely the Clifford group, ${\cal C}_n$, defined as a
subset of unitary operators that normalise the Pauli group *INSERT CLIFFORD GROUP DEFINITION*.
Notable elements of this group are the Hadamard, Controlled-Not and Phase operators.

The Hadamard operator, $H$ maps computational basis states to a superposition of computational basis states, written explicitly 
in it's action as, 
\begin{align*}
    H|0\rangle = \frac{|0\rangle + |1\rangle}{\sqrt{2}}, && H|1\rangle = \frac{|0\rangle - |1\rangle}{\sqrt{2}},
\end{align*}
or in matrix form, 
\begin{equation}
    H = \frac{1}{\sqrt{2}} \begin{bmatrix}
        1 & 1\\
        1 & -1
    \end{bmatrix}.
\end{equation}

Controlled-NOT, $CNOT_{AB}$, is a controlled two-qubit gate. The first qubit, $A$ acts as a `control' for an operation to be 
performed on the target qubit, $B$. It's matrix representation is, 
\begin{align*}
    CNOT_{12} = \begin{bmatrix}
        1 & 0 & 0 & 0 \\
        0 & 1 & 0 & 0 \\
        0 & 0 & 0 & 1 \\
        0 & 0 & 1 & 0
        \end{bmatrix}
\end{align*}

The Phase operator, denoted $R$ is defined as,
\begin{align*}
    R =
    \begin{bmatrix}
        1 & 0                  \\
        0 & e^{i\frac{\pi}{2}}
    \end{bmatrix}.
\end{align*}

\subsection{Entanglement in Qubit Systems}

The CNOT operator is often used to an generate entangled state. One such state is the maximally entangled 2-qubit state,
called a Bell state, $|{\bm\Phi}^{+}\rangle_ = (|00\rangle + |11\rangle)/\sqrt{2}$. This is prepared
from a $|00\rangle$ state, by applying a Hadamard to the first qubit, and subsequently a Controlled-Not gate:
\begin{itemize}
    \item[I.] $H \otimes I |00\rangle = \left (\frac{|0\rangle + |1\rangle }{\sqrt{2}}\right )|0\rangle$ 
    \item[II.]  $CNOT \left (\frac{|0\rangle + |1\rangle }{\sqrt{2}}\right )|0\rangle = \frac{|00\rangle + |11\rangle}{\sqrt{2}}$
\end{itemize}
% \begin{align}
%     H \otimes I |00\rangle = \left (\frac{|0\rangle + |1\rangle }{\sqrt{2}}\right )|0\rangle \\
%     CNOT \left (\frac{|0\rangle + |1\rangle }{\sqrt{2}}\right )|0\rangle = \frac{|00\rangle + |11\rangle}{\sqrt{2}}
% \end{align}
The corresponding circuit representation of this preparation is given in Fig. \ref{Bellstate}.

\begin{figure}
    \centering
    \begin{quantikz}
        \lstick{$\ket{0}$} & \gate{H} & \ctrl{1} & \qw\rstick[wires=2]{$\ket{\Phi^+}$} \\
        \lstick{$\ket{0}$}& \qw & \targ{} & \qw
    \end{quantikz}
    \caption{Preparation of a Bell state from $\ket{0}$ using a Hadamard and CNOT.}
    \label{Bellstate}
\end{figure}
The output Bell state, cannot be written in product form. That is, the state cannot be written as,
\begin{align*}
    |{\bm\Phi}^+\rangle = & \left[ \alpha_0 |0\rangle + \beta_0|1\rangle\right] \otimes \left[\alpha_1 |0\rangle + \beta_1|1\rangle\right] \\
    =                     & \alpha_0\beta_0 |00\rangle + \alpha_0\beta_1|01\rangle + \alpha_1\beta_0|10\rangle + \alpha_1\beta_1|11\rangle
\end{align*}
since the $\alpha_0$ or $\beta_1$ must be zero in order to ensure the $|01\rangle$, $|10\rangle$ vanish.
However, this would make the coefficients of the $|00\rangle$
or $|11\rangle$ terms zero, breaking the equality. Thus, $|{\bm\Phi}^+\rangle$ cannot be written in
product form and is said to be entangled. This defines a general condition for a arbitrary state to be entangled 
\cite{nielsen_chuang_2010}.


% To give an example, consider the preparation of a GHZ state, $\frac{|000\rangle + |111\rangle}{\sqrt{2}}$ 
% from an initial all-zero state, $|000\rangle$. This transformation may be written as, 
% \begin{align*}
%    |\psi_{GHZ}\rangle = U |000\rangle,\\
%    |\psi_{GHZ}\rangle = (CNOT_{13})(CNOT_{12})(H\otimes I \otimes I)|000\rangle. 
% \end{align*}
% Where $I$ is the identity operator. As a circuit diagram, the transformation takes the following form: 
%    \begin{center}
%     \begin{quantikz}
%         \lstick{$\ket{0}$} & \gate{H} & \ctrl{1} & \ctrl{2} & \qw\rstick[wires=3]{$\ket{\psi_{GHZ}}$} \\
%         \lstick{$\ket{0}$}& \qw & \targ{} & \qw & \qw\\
%         \lstick{$\ket{0}$}& \qw & \qw & \targ{} & \qw
%     \end{quantikz}
%     % \caption{Preparation of a Bell state from $\ket{0}$ using a Hadamard and CNOT.}
% \end{center}

%%%%%%%%%%%%%%%






%can perform operations on qubits in a similar fashion to logical operations on bits, since maps are linear we can express them as matrices. introduce some operators. 
%then explain how we may form a quantum circuit from these. 
%then explain how we encode information



\section{Fermionic Systems}


%Indistinguishibility. 

The familiar qubit system may be mapped onto a system of identical particles (fermions), such that the 
overall many body state describing the system, is invariant under particle exchange. This is 
performed via a Jordan-Wigner transformation, which maps any local spin-model to a local fermionic model \cite{10.1093/acprof:oso/9780199573127.001.0001}.
To gain an understanding of how this can be carried out and why it is relevant, it will be useful to introduce the core concepts and language from \textit{Second Quantization}.
\subsubsection{Second Quantization and Indistinguishable Particles}
The wave function for a system of $N$ identical particles is $\psi(x_1, x_2, \dots x_N)$, where a particle is specified by it's position vector, $\vec{x}$. For bosonic systems, the wavefunction is symmetric under particle exhchange, while fermionic systems present anti-symmetric wavefunctions under particle exchange called Slater determinants. 

The construction of the $n$ particle state via the extension of the single particle wavefunction, as described in (\refeq{nqubit}), 
leads to a redundancy in it's description of a many-body state and an unnecessarily large Hilbert space.
A more efficient appraoch to describing many-body states is found in the formalism of second quantisation. Instead of describing states with Slater determinants, a \textit{Fock state} presents an elegant and convinient basis in which to work in. The Fock state of a many-body system is represented in an occupancy number basis, written as $|n_1, n_2, \dots, n_L\rangle$, where $n_i$ is the occupation number of a given local fermionic mode (LFM). For systems of bosonic particles,  the occupancy can take any real non-negative integer. For systems of many fermionic particles, the occupancy number is either $0$ or $1$ with no two fermions ever occupying the same mode.

To preserve the symmetric properties of Fock states, second quantization introduces fermionic creation and annihilation operators. The creation operator, $a^{\dagger}_{i}$ creates a particle at site $i$ if unoccupied, and the annihilation operator, $a_i$, removes a particle at site $i$ if occupied. More formally, it's action on a Fock state may be written as, 
\begin{align}
    a_j^{\dagger} |n_0, \dots, n_j, \dots, n_{m-1}\rangle = {(-1)}^{\sum^{j-1}_{s=0}n_s} (1-n_j) |n_0, \dots, n_j - 1, \dots, n_{m-1}\rangle, \\
    a_j |n_0, \dots, n_j, \dots, n_{m-1}\rangle = {(-1)}^{\sum^{j-1}_{s=0}n_s} n_j |n_0, \dots, n_j - 1, \dots, n_{m-1}\rangle.
\end{align}
With the fermionic creation and annihilation operators obeying crucial anti-commutation relations:
\begin{align}
    \{a_j, a_k\} \equiv \{a_j^{\dagger}, a_k^{\dagger}\} = 0, && \{a_j, a_k^{\dagger}\} = \delta_{jk}I.
\end{align}

Using these operators, an arbitrary Fock state, $|\psi_{F}\rangle$ of $L$ modes, may be prepared from a set of creation operators acting on the vacuum, $|\Omega\rangle = |0, 0 \dots, 0\rangle$, 
\begin{equation}
    |\psi_{f}\rangle = (a_{1}^{\dagger})^{n_1} (a_{2}^{\dagger})^{n_2} \dots (a_{L}^{\dagger})^{n_{L}}|{\Omega}\rangle.
\end{equation}

\subsubsection{Jordan-Wigner Transformation}

Using the fermionic operators, the Jordan-Wigner transformation may be defined, which maps a system of $N$ spin-$\frac{1}{2}$ particles onto a system of $N$ `spinless' fermions by defining the fermionic operators in terms of the Pauli spin operators \cite{landahl2023logical}, 
\begin{align}
    a^{\dagger}_i = \left(\prod^{i-1}_{k = 1} Z_k\right) \sigma^{-}_{i}, &&
    a_i = \left(\prod^{i-1}_{k = 1} Z_k\right) \sigma^{+}_{i}.
\end{align}
Where $\sigma^{\pm}_{j} = (X_j \pm i Y_j)/2$.
Such a mapping allows the Hilbert space of $N$ qubits to be identified with the Hilbert space of $N$ local fermionic modes (LFM's). However this transformation is non-local, with fermionic operators having support over the almost the entire Hilbert space of $N$ qubits \cite{ Ba_uls_2007}. 


As an alternative description, the \textit{Majorana operators} may be defined as 
$c_{2k} = a_k + a_k^{\dagger}$ and $c_{2k+1} = i(a_k^{\dagger} - a_{k})$ for $k =  1, \dots, N$ such that they satisfy $\{c_j, c_k\} = 2\delta_{jk}$. This description splits each LFM in two, increasing the number of modes to $2N$. 





\section{Scrambling Dynamics}
\subsection{Random Unitary Circuits}
To study the generic quantum many-body systems, random unitary circuits provide a minimally structured model that can emulate the required dynamics of generic or `chaotic' unitary evolutions.\cite{https://doi.org/10.48550/arxiv.2210.10129}.
In the study of operator hydrodynamics The set-up is a chain of $L$ spins, labelled as $q = 0, 1, \dots, L-1$ each with their respective local dimension $h$ and is subject to a random unitary circuit $\mathcal{U}$. The total circuit is constructed per timestep, by acting with a layer of 2-qubit unitaries on evenly-bonded spins, followed by a layer of unitaries applied to odd-bonded spins. More formally, each timestep in the circuit corresponds to an action by $U = U_{\text{even}} U_{\text{odd}}$, where $U_{\text{even}}$ is given by the tensor product of individual 2-qubit unitaries $U_{q, q+1}$ applied to all evenly bonded sites, $q, q+1$ for $q$ even,  and $U_{\text{odd}}$ is given by the tensor product of individual 2-qubit unitaries $U_{q, q+1}$ applied to all odd bonded sites, $q, q+1$ for $q$ odd. Each 2-qubit unitary acts on nearest neighbour spins, and is drawn from the uniform (Haar) probability distribution on the unitary group $U(4)$ or from a specific subgroup \cite{hunterjones2018operator}.
For our purposes, we require a simpler model. For a system of $L$ qubits, each time step corresponds to a randomly drawn unitary acting on a randomly drawn qubit, $q_i$ and it's nearest neighbour, $q_{i+1}$.

% such as the Clifford group, $\mathcal{C} \equiv \{H, CNOT, R\}$.

To extract solvable dynamics and utilise well-estabilished measures, information about the state of the system is encoded as strings of operators. A convinient basis to work with in qubit systems is found in the set of Pauli matrices.
For example, a system with $L = 5$ spins could be represented by the operator string $\mathcal{O} = I \otimes I \otimes P_i \otimes I \otimes I$, where $P_i$ is an arbitrary Pauli matrix. An operator of this form is a \textit{local} operator, as it only acts non-trivially on a single site. 
For a system subject to generic unitary evolution , we expect that information encoded by an initially simple product operator to spread over the large number of degrees of freedom becoming highly complicated sum of global operators, such that they have support over the entire system. This process is known as Quantum Scrambling, and can be regarded as the combined notion of \textit{Operator Spreading} and a growth in \textit{Operator complexity}. 

\subsection{Operator Spreading} 
To give a picture of operator spreading, this section will primarily focus on the evolution of Pauli strings. Starting from some initially local product operator, such as 
$\mathcal{O} = \mathbb{1} \otimes\dots \otimes \mathbb{1} \otimes \sigma_{x} \otimes \mathbb{1} \otimes \dots \otimes \mathbb{1}$. This system will evolve via $|\psi (t)\rangle = U |\psi_0\rangle$ and the operator string will evolve via the Heisenberg evolution of operators, 
\begin{equation}\label{heisenberg_evolution}
    \mathcal{O}(t) = U(t) \mathcal{O} U^{\dagger}(t).
\end{equation}
 Following this evolution, the local operator with minimal support, $O_j$ has evolved to $O_j(t)$ with support over a
large region of sites \cite{Khemani_2018}. Operators that grow in this way, will spread ballistically \cite{Roberts_2015, Lieb:1972wy, Schuster_2022} and are often
characterised by the out-of-time ordered correlator (OTOC) \cite{Xu2022} and the square-commutator\cite{Blake_2018}. This also gives an
intuitive picture of operator spreading, with the squared commutator defined as 
\begin{equation}
  C(t) = \langle [O(t), V_i][O(t), V_i]^{\dagger}\rangle
\end{equation}
where $V_i$ is a static local operator at site $i$ \cite{https://doi.org/10.48550/arxiv.1804.08655}. Then at $t=0$, $O(t)$
acts on a single site or a finite region of sites, such that it commutes with the static operator, $V_i$ and $C(t) = 0$.
Once the operator spreads, and becomes more non-local, the commutator increases as it's support overlaps with $V_i$.


\subsection{Operator Complexity and Entanglement Entropy}
A key component in quantum supremacy lies in a quantum algorithm's ability to utilise the abudant \textit{entanglement} within a system. In simple bipartite systems, where $\mathcal{H}_{AB} = \mathcal{H}_{A}\otimes \mathcal{H}_{B}$, the state of this system, $|\Psi_{AB}\rangle \in \mathcal{H}_{AB}$ is 
said to be entangled if and only if the state cannot be written in product form, $|\Psi_{AB}\rangle =  |\Psi_{A}\rangle \otimes |\Psi_{B}\rangle$, where $|\Psi_{A}\rangle$ and $|\Psi_{B}\rangle$ are the two vectors corresponding to the Hilbert spaces of each subsystem. If the state $|\Psi_{AB}\rangle$ is not entangled, then it is a product state and is said to be separable \cite{Horodecki_2009}.
 

\subsubsection{Entanglement In Qubit Systems}
Qubit systems provide the simplest decription of entangled states in bipartite systems. For a system of two qubits, there exist 4 specific maximally entangled configurations, called Bell states: 

\begin{align}\label{bellpair}
    |\Phi^{+}\rangle = \frac{|00\rangle + |11\rangle }{\sqrt{2}}, && |\Phi^{-}\rangle = \frac{|00\rangle - |11\rangle }{\sqrt{2}},
    \\
    |\Psi^{+}\rangle = \frac{|01\rangle + |10\rangle }{\sqrt{2}}, && |\Psi^{-}\rangle = \frac{|01\rangle - |10\rangle }{\sqrt{2}}.
\end{align}
We can verify that the state, $|\Phi^{+}\rangle $ is an entangled state by writing, 
\begin{align}\label{entangle}
    |{\Phi}^+\rangle &=  \left[ \alpha_0 |0\rangle + \beta_0|1\rangle\right] \otimes \left[\alpha_1 |0\rangle + \beta_1|1\rangle\right], \\
    &=                      \alpha_0\beta_0 |00\rangle + \alpha_0\beta_1|01\rangle + \alpha_1\beta_0|10\rangle + \alpha_1\beta_1|11\rangle.
\end{align}
We require that the $\alpha_0$ or $\beta_1$ terms must be zero in order to ensure the $|01\rangle$, $|10\rangle$ vanish. However, this would make the coefficients of the $|00\rangle$ or $|11\rangle$ terms zero, breaking the equality. Thus, $|{\bm\Phi}^+\rangle$ cannot be written in product form and is said to be entangled.

\subsubsection{Entanglement In Fermionic Systems}

For a system of indistinguishable particles, each with only one spin degree of freedom (effectively spinless), entanglement can emerge as states analgous to a Bell state (\refeq{bellpair}). Considering a system of 2 modes, occupied by a single indistinguishable particle, a fermion. The state of this system can be expressed as, 
\begin{equation}
    |\psi_{f}\rangle = \frac{|0_{A}\rangle |{1_{B}}\rangle + |1_{A}\rangle |0_{B}\rangle}{\sqrt{2}}.
\end{equation}
Where $A, B$ `label' the two respective modes, and are included to express the state of this system intuitively.
One can easily verify that this state is entangled, as outlined in (\refeq{entangle}), as it cannot be expressed in product form.  However, this definition of entanglement is of little use when working with larger multi-fermion systems. We require entanglement measures to make qualitative statements on the complexity of operators and the systems they represent \cite{Eckert_2002}.

\subsubsection{Entanglement Measures}

For many-body systems, the definition of entanglement cannot be directly used to detect entanglement between states. Instead, entanglement measures provide a useful tool for detecting and analysing entanglement dynamics. In the case of distinguishable particles, one such entanglement measure is the \textit{Schmidt Rank}, derived from the \textit{Schmidt Decomposition}. The Schmidt decomposition states for any state vector, $|\Psi\rangle \in \mathcal{H}$ where $\mathcal{H} = \mathcal{H}_{A} \otimes \mathcal{H}_{B}$, there exists an orthonormal basis $\{|a_i\rangle \otimes |b_j\rangle\}$ such that,
\begin{equation}\label{Schmidt}
    |\Psi\rangle = \sum_{i = 1}^{d_A} \sum_{j = 1}^{d_B} D_{ij} |a_i\rangle \otimes |b_j\rangle.
\end{equation}
Where $d_A$ and $d_B$ are the dimensions of the respective subsystems, $A$ and $B$. The matrix $D$ is a matrix of coefficients with the rank being the Schmidt rank. If the Schmidt rank is of rank one, then the state $|\Psi\rangle$ is a product state \cite{Horodecki_2009}. 

In systems of indistinguishable particles, a similar measure exists in the form of the \textit{Slater Rank criterion}. The Slater Decomposition states that for a system of two fermions, in a $N$-dimensional  space specified by the state vector,
\begin{equation}\label{Slater}
    |\Psi\rangle = \sum_{i, j = 1}^N \omega_{ij} a^{\dagger}_{i}a^{\dagger}_j|\Omega\rangle.
\end{equation}
There exists a unitary transformation, $U$ that block diagonalises the coefficient matrix $\omega$. More explicitly,
\begin{align*}
    \omega' = U\omega U^{T} = \text{diag}[Z_0, Z_1, \dots, Z_r], && Z_i = 
    \begin{bmatrix}
        0 & z_i \\
        -z_i & 0
    \end{bmatrix}.
\end{align*}
Where $z_i >0$ $ \forall i \in 1, \dots, r$ and $Z_0$ is an all-zero matrix of size $(N \times 2r) \times (N - 2r)$. The Slater rank is then the number of non-vanishing block matrices \cite{Schliemann_2001}.

Entanglement measures such as the Schmidt and Slater rank, provide useful criterion for entanglement amongst state vectors. However, our aim is to analyse operator complexity signified by the entanglement in operator space. Thus, entanglement measures such as (\refeq{Schmidt}) and (\refeq{Schmidt}), do not provide the adequate tools to analyse the dynamics operator complexity in exactly solvable models. Instead we utilise an entropy entanglement measure, known as the Von Neumann entropy. For a quantum system described by the density matrix, $\rho$, the the Von Neumann entropy, $S$, is defined in general as, 
\begin{equation}
    S = -\text{Tr}(\rho \ln \rho).
\end{equation}
To determine the entanglement entropy, a bipartition is introduced into the system, so that the entanglement entropy describes to which degree the two subsystems that have been partition are entangled. The entanglement entropy is then, 
\begin{equation}\label{entrop}
    S_A = -\text{Tr}(\rho_A \ln \rho_A) = -\text{Tr}(\rho_B \ln \rho_B).
\end{equation}
Where $\rho_A$ , $\rho_B$ describe the subsystems, $A, B$ \cite{PhysRevA.51.2738}.





% This measure provides a nice introduction to how entanglement measures are constructed in bipartite systems, but are not fit for the purposes of this project. A full description can be found in Appendix (X). 
% For bipartite systems of distinguishable particles, the Schmidt rank, derived from the Schmidt decomposition provides a method in determining whether a many-body state is entangled. 











% \begin{figure}
    \centering
    % \begin{quantikz}[transparent]
    %     \lstick{$q_0$}& \gate[2]{U_{01}} & \qw & \qw & \qw \\
    %     \lstick{$q_1$}& & \gate[2]{U_{12}} & \qw & \qw \\
    %     \lstick{$q_2$}& \gate[2]{U_{23}} &&& \qw\\
    %     \lstick{$q_3$}& \qw &&& \qw
    % \end{quantikz}
    \begin{tikzpicture}
        
    \end{tikzpicture}

    \caption{Preparation of a Bell state from $\ket{0}$ using a Hadamard and CNOT.}
    \label{randomunitarycircuit}
\end{figure}
% Quantum scrambling is the process in which local information encoded by a simple product operator is rapidly spread over a large number of degrees of freedom 

% The information becomes highly non-local and the initial simple product operator becomes a complicated sum of product operators. 

% \section{Results}

\newpage
\section{Simulating a Quantum Circuit}
\subsection{Stabilizer Circuits}
\subsection{Blake and Linden's Construction}
\subsection{Free Fermion Circuits}

\section{Results}
\subsection{Stabilizer Circuits}
\subsection{Non-interacting Fermion Circuits}
\subsubsection{Non-Number Conserving Gates}
\subsubsection{Number-Conserving Gates}

\section{Discussion}

\section{Conclusion}



\bibliographystyle{ieeetr}
\bibliography{report_bibliography.bib}



\end{document}