\section{Classically Simulable Quantum Circuits}
%Maybe some introduction, short sentence on what classical simulability is.
\subsection{Clifford Circuits and Stabilizer Formalism}

Quantum circuits comprised of only quantum gates from the Clifford group are known as \textit{Clifford Circuits}.
This family of circuits are of considerable interest due to their classical simulability and ability to
showcase complex phenomena \cite{doi:10.1063/5.0054863}, that is,
circuits of this construction can be efficiently simulated on a classical computer with polynomial
effort via the Gottesman-Knill theorem. For this reason, Clifford gates do not form
a set of universal quantum gates, meaning a universal quantum computer cannot be constructed using only
Clifford unitaries. A set of universal quantum gates allows for any unitary operation to be approximated to
arbitrary accuracy by a quantum circuit, constructed using only the original set of gates.

\subsubsection{Stabilizer Formalism}
An arbitrary, pure quantum state, $|\psi\rangle$ is \textit{stabilized} by
a unitary operator, $M$ if $|\psi\rangle$  is an eigenvector of $M$, with eigenvalue 1, satisfying
\begin{equation}
    M|\psi\rangle = |\psi\rangle.
\end{equation}
It will be convinient to utilise the language of stabilizers, to define an initial state by the operators that stabilize it
\cite{PhysRevA.70.052328}.
This can be seen from considering the Pauli matrices, and the unique states they stabilize. In the one-qubit case these are
the $+1$ eigenstates of the pauli matrices (omitting normalisation factors)
\begin{align}
     & X(|0\rangle + |1\rangle) = |0\rangle + |1\rangle,   \\
     & Y(|0\rangle + i|1\rangle) = |0\rangle + i|1\rangle, \\
     & Z|0\rangle = |0\rangle.
\end{align}

If given a group or subgroup of unitaries, $\cal U$, the vector space, $V_n$, of $n$ qubit states is
stabilized by $\cal U$ if every element of $V_n$ is stable under action from every element of $\cal U$.
This description is more appealing, as we can exploit mathematical techniques from group theory to
describe quantum states and vector spaces. In particular, a group $\cal U$ can be described using it's
generators. In general, a set of elements  $g_1, \dots, g_d$ of a group, $\cal G$, generate the group
if every element of $\cal G$ can be written as a product of elements from the set of generators,
$g_1, \dots, g_d$, such that ${\cal G}$ is expressed as ${\cal G} \coloneqq \langle g_1, \dots, g_d\rangle$ \cite{Calderbank_1997}.
%  For example,
% consider the group ${\cal U} \equiv \{ I, Z_1Z_2, Z_2Z_3, Z_1Z_3\}$. ${\cal U}$ can be compactly written
% as ${\cal U} = \langle Z_1Z_2, Z_2Z_3 \rangle$, by recognising $(Z_1Z_2)^2 = I$ and
% $Z_1Z_3 = (Z_1Z_2)(Z_2Z_3)$.

This allows for the description of a quantum state, and subsequently it's dynamics, in terms of the generators
of a stabilizer group. To see how the dynamics of a state are represented in terms of generators, consider
a stabilizer state under the action of an arbitrary unitary operator:
\begin{align}
    UM|\psi \rangle =  UMU^{\dagger}U|\psi\rangle
\end{align}
The state $|\psi \rangle$ is an eigenvector of $M$ if and only if, $U |\psi\rangle$ is an
eigenvector of $UMU^{\dagger}$. Thus, the application of an unitary operator transforms
$M \to UMU^{\dagger}$. Moreover, if the state $|\psi \rangle$ is stabilized by $M$, then the evolved
state $U|\psi\rangle$ will be stabilized by $UMU^{\dagger}$. If $M$ is an element of a stabilizer group
$\cal S$ such that $M_1, \dots, M_l$ generate $\cal S$, then $UM_1U^{\dagger}, \dots, UM_lU^{\dagger}$ must generate
$U{\cal S}U^{\dagger}$. This implies that to compute the dynamics of a stabilizer, only the transformation of
the generators needs to be considered \cite{fault-tolerantQC}. It is because of this,
Clifford circuits are able to be efficiently classically simulated via the Gottesman-Knill theorem:
\begin{theorem}[Gottesman-Knill Theorem \cite{knillGottesman}] *INSERT EXPLANATION*

    Given an $n$ qubit state $|\psi \rangle$, the following statements are equivalent:
    \begin{itemize}
        \item $|\psi \rangle$ can be obtained from $|0 \rangle^{\otimes n}$ by CNOT, Hadamard and phase gates only.
        \item $|\psi \rangle$ is stabilized by exactly $2^n$ Pauli operators
        \item $|\psi \rangle$ can be uniquely identified by the group of Pauli operators that
              stabilize $|\psi \rangle$.
    \end{itemize}
\end{theorem}
Evolution via unitaries from the Clifford group, take an appealing form when acting on simple Pauli operators that act as stabilizers.
In particular, Clifford unitaries map Pauli operators to products of Pauli operators, and since any product of Pauli operators
can be written in terms of only products of $X$ and $Z$, a generic Pauli string may be written as,
\begin{equation}
    S \propto X_1^{v_{1_x}}Z^{v_{1_z}}_1 \dots X_N^{v_{N_x}}Z^{v_{N_z}}_N,
\end{equation}
and can be specified by a binary vector,
\begin{equation}
    \vec{v} = (v_{1_x},  v_{1_z}, \dots, v_{N_x}, v_{N_z}).
\end{equation}
Thus, conjugation by Clifford unitaries on $S$, corresponds to local updates in $\vec{v}$.

The operation of Clifford unitaries on Pauli operators by conjuagtion is shown in Table \ref{table:1}.
\begin{table}[h!]
    \centering
    \begin{tabular}{ ||c|c|| }
        \hline
        Clifford Unitary        & Operation        \\
        \hline\hline
        \multirow{4}{3em}{CNOT} & $X_1 \to X_1X_2$ \\
                                & $X_2 \to X_2$    \\
                                & $Z_1 \to Z_1$    \\
                                & $Z_2 \to Z_1Z_2$ \\
        \hline
        \multirow{2}{1em}{H}    & $X \to Z$        \\
                                & $Z\to X$         \\
        \hline
        \multirow{2}{0.9em}{S}  & $X \to X$        \\
                                & $Z \to -Z$       \\
        \hline
    \end{tabular}
    \caption{Action of Unitary operators from the Clifford group on Pauli operators by conjuagation.}
    \label{table:1}
\end{table}


\subsection{Non-interacting Fermi Circuits}
\subsubsection{Fermionic Fock Space}
Another class of circuits that are classically simulable, are known as non-interacting or free fermi circuits.
These circuits are derived from a fermionic Fock space formalism, where the basic units of information are local
fermionic modes (LFMs). Each mode can be occupied or unoccupied in the same fashion as qubits, where
a mode $j$ is associated with an occupation number, $n_j = 0$ (unoccupied mode) or $n_j = 1$ (occupied mode).
The state space of a many-fermion system is the fermionic Fock space, ${\cal F} = {\cal H}_0 \oplus {\cal H}_1$,
where ${\cal H}_0$ and ${\cal H}_1$ correspond to subspaces of even and odd number of particles respectively.
The natural basis vectors for a Fock space of $m$ LFMs are Fock states $|n_0, \dots, n_{m-1}\rangle$,
where each element is the previously mentioned occupation number, $n_j$ at each site \cite{DeFelice2019}.
This system is similar to the qubit set-up as the Hilbert space for $m$ LFMs is identified with the Hilbert space
of $m$ qubits such that
\begin{align*}
    |n_0, \dots, n_{m-1}\rangle \equiv | n_0 \rangle \otimes | n_1 \rangle \otimes \dots \otimes | n_{m-1} \rangle, &  & n_j = 0,1
\end{align*}
where the RHS of the expression descibes a system of $m$ qubits. This allows a mapping of a qubit system (spin 1/2 system) onto
a fermionic system via a Jordan-Wigner Transformation, such that the representation and dynamics of a quantum
state are described using creation and annihilation operators from second quantization.
The creation operator, $a_j^{\dagger}$ creates a fermion at mode $j$ (if the mode is unoccupied),
whilst the annihilation operator, $a_j$ removes a fermion at site $j$ if the mode is occupied.
In the case of fermions, these operators obey canonical anti-commutation relations:
\begin{align}
    \{a_i, a_j\} \equiv \{a_i^{\dagger}, a_j^{\dagger}\} = 0, &  & \{a_i, a_j^{\dagger}\} = \delta_{ij}I,
\end{align}
The occupation number, $n_j$ is then the eigenvalue of the number operator, $\hat{n}_j = a^{\dagger}_j a_j$
The creation and annihilation operators act on Fock states in the following way:
\begin{widetext}
    \begin{align*}
        a_j^{\dagger} |n_0, \dots, n_j, \dots, n_{m-1}\rangle = (-1)^{\sum^{j-1}_{s=0}n_s}\delta_{n_j, 0}  |n_0, \dots, 1, \dots, n_{m-1}\rangle, \\
        a_j |n_0, \dots, n_j, \dots, n_{m-1}\rangle = (-1)^{\sum^{j-1}_{s=0}n_s}\delta_{n_j, 1}  |n_0, \dots, 0, \dots, n_{m-1}\rangle,
    \end{align*}
\end{widetext}

To prepare an arbitrary Fock state, $|{\mathbf n}\rangle$, from a vacuum, $|{\mathbf 0}\rangle = |0,0,\dots,0\rangle$,
with fermions occupying arbitrary positions, creation operators are labeled to act on specified modes:

\begin{equation}
    |{\mathbf n}\rangle = (a_{1}^{\dagger})^{n_1} (a_{2}^{\dagger})^{n_2} \dots (a_{m-1}^{\dagger})^{n_{m-1}}|{\mathbf 0}\rangle
\end{equation}

\subsubsection{Fermionic Circuits}
A fermionic circuit is constructed from a sequence of elementary gates that mediate an interaction between
modes $j$ and $k$, written as $U = e^{iH_g}$. Where $H_g$ is a general gate Hamiltonian: *REVIEW*
\begin{equation}\label{gateham}
    H_g = b_{jj}a_j^{\dagger}a_j + b_{kk} a_{k}^{\dagger}a_{k} + b^{*}_{jk}a_k^{\dagger}a_j,
\end{equation}
where $b_{jj}, b_{jk}, b_{kk}$ are complex coefficients that form a hermitian matrix, $\mathbf b$.
The circuit is constructed by defining the
sequence of gates as $U = U_{(n)} \dots U_2 U_1$, such that the number of gates, $n$ is no more than
polynomial in $n$ in order to be simulated classically. Then considering the evolution of Fock state in a circuit, given that the
circuit preserves the number of fermions ($U|{\mathbf 0}\rangle = |{\mathbf 0}\rangle  $), the state evolves as:
\begin{align}
    U a_j^{\dagger} |{\mathbf 0}\rangle = U a_j^{\dagger} U^{\dagger}U |{\mathbf 0}\rangle = U a_j^{\dagger} U^{\dagger} |{\mathbf 0}\rangle
\end{align}
Where $U$ acts by conjugation as
\begin{equation}
    U a_j^{\dagger} U^{\dagger} = \sum_s B_{is} a_s^{\dagger}
\end{equation}
and the matrix $B = \exp(i{\mathbf b})$ is defined from the previous coefficient matrix in Eq. (\refeq{gateham}).
To compute the final state of the system, only the matrix $B$ needs evaluation, which is classically simulable if $U$ contains
a polynomial number of gates \cite{Terhal2001}. More generally, if $U$ acts on an arbitrary Fock state, $|{\mathbf n}\rangle$, it's action
is written as:
\begin{align}
    U |{\mathbf n}\rangle & = U a_{i_1}^{\dagger} a_{i_2}^{\dagger}\dots a_{i_m}^{\dagger}|{\mathbf 0}\rangle                                                    \\
                          & = \sum_{j_1 \dots j_k} B_{i_1, j_1} \dots B_{i_k, j_k} a_{j_1}^{\dagger} a_{j_2}^{\dagger}\dots a_{j_k}^{\dagger}|{\mathbf 0}\rangle
\end{align}
Some simpler operators that act on Fock states, are the spin 1/2 Pauli operators, that are mapped onto a
non-interacting fermion system via the Jordan Wigner Transformation (JWT).
In particular, the set of Pauli operators that act on the $j$th mode,
$\{X_j, Y_j, Z_j\}$, can be expressed in terms of the creation and annihlation operators by first defining
the operators $\sigma^{\pm}_j = \frac{1}{2}(X_j \pm iY_j)$. Then in terms of the creation and annihilation
operators:
\begin{align}
    \sigma^{+}_j = e^{i\pi\sum^{j-1}_{m = 0} a^{\dagger}_m a_m} a_j^{\dagger} \\
    \sigma^{-}_j = e^{i\pi\sum^{j-1}_{m = 0} a^{\dagger}_m a_m} a_j
\end{align}
The inverse transformation can also be made, to define the Majorana
fermion operators that effectively split each LFM into halves \cite{Bravyi2000};
\begin{align}
    c_{2j} = a_j + a_j^{\dagger} &  & c_{2j+1} = \frac{a_j - a_j^{\dagger}}{i}
\end{align}
Majorana fermion operators satisfy the anti-commutation relation;
\begin{equation}
    \{c_j, c_k\} = 2\delta_{jk}
\end{equation}
and therefore can be expressed in terms of Pauli operators,
\begin{align}
    c_{2j} = X_j \prod^{j-1}_{k=0} Z_k &  & c_{2j + 1} = Y_j \prod^{j-1}_{k=0} Z_k
\end{align}
where the term $\prod^{j-1}_{k=0} Z_k$ is called a Jordan Wigner string.













