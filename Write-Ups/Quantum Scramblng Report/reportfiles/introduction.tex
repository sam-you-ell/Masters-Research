

\section{Introduction}
Many-body systems and their dynamics play a central role in our understanding of modern physics with the
dynamics of quantum many-body systems offering a great insight to a wide variety of fields in contemporary physics. Studying the evolution of quantum many-body systems presents challenging problems and has provided insightful results in condensed matter physics and quantum information \cite{Polkovnikov_2011}.
Studying the dynamics of such systems is fundamentally
a computational challenge due to the exponential growth of the Hilbert space with the number of
qubits. However, quantum dynamics can be efficiently simulated in atypical quantum circuit models,
providing a rich theoretical playground to test our understanding and uncover interesting phenomena.
These atypical circuit models, known as classically simulable circuit models, 
will be analysed in hopes of understanding the spreading and scrambling of encoded local information,
a process known as quantum scrambling. More precisely, quantum scrambling describes the
process in which local information encoded by a simple product operator, such as a string of Pauli operators, becomes 'scrambled' by a  amongst the large number of degrees of freedom in the system, such that, the operator becomes a highly complicated sum of product operators.

In recent years, the study of this scrambling process has seen immense research yielded new perspectives. These include new insights into the study of information in black holes and the 
AdS/CFT correspondence \cite{Calabrese_2009,Jensen_2016,Sekino_2008,ShenkerBlackHolesButterfly2014},
quantum chaos, \cite{Maldacena_2016, https://doi.org/10.48550/arxiv.1412.6087} and operator hydrodynamics in many-body systems \cite{Khemani_2018, PhysRevX.8.021013, PhysRevX.8.031058,Grozdanov_2018}.

In order to investigate this phenomenon,classically simulable models provide excellent tools to gain a deeper understanding of operator evolution in hopes of applying any findings
to generic cases of unitary evolution in many-body systems.
As previously mentioned, some families of circuits are able to be efficiently
simulated with polynomial effort on a classical computer. Two well-studied circuit models take the principal interest of this review, namely Clifford circuits and non-interacting fermion circuits. Clifford circuits and the Clifford
group have played a key role in the study fault-tolerant quantum computing and, more recently, many-body physics via
random Clifford unitaries \cite{PhysRevB.98.205136, https://doi.org/10.48550/arxiv.2110.02988}.


This is due to their efficient simulability on classical computers, despite the high amounts of entanglement that can be generated amongst states. Hence, circuits such as Clifford circuit are an ideal candidate in the study of many-body systems and their dynamics \cite{https://doi.org/10.48550/arxiv.2210.10129}.

The second class of circuit models to be explored, are Matchgate or non-interacting fermion circuits. First introduced by Valiant \cite{Valiant2001QuantumCT} in the matchgate formalism and later identified with a physical model of non-interacting fermions in one dimension. This system has been shown to be classically simulable by DiVincenzo and Terhal \cite{Terhal2001} as an extension of Valiant's findings. Thus, free fermion circuits form a good candidate to study scrambling phenomena, and the limitations of these circuit models.

With the use of recently developed techniques, the key objective of this project is to examine 
operator scrambling in models where the dynamics are analytically and numerically tractable. This is not
possible in the generic unitary evolution of many-body systems due to an exponential growth of
the Hilbert space with system size. However, by simulating the dynamics of scrambling in simulable circuit models, a picture of unitary evolution can be constructed, in hopes of gaining an understanding of dynamics of operator evolution and entanglement in atypical quantum systems. 





